% Options for packages loaded elsewhere
\PassOptionsToPackage{unicode}{hyperref}
\PassOptionsToPackage{hyphens}{url}
%
\documentclass[
]{article}
\usepackage{amsmath,amssymb}
\usepackage{lmodern}
\usepackage{iftex}
\ifPDFTeX
  \usepackage[T1]{fontenc}
  \usepackage[utf8]{inputenc}
  \usepackage{textcomp} % provide euro and other symbols
\else % if luatex or xetex
  \usepackage{unicode-math}
  \defaultfontfeatures{Scale=MatchLowercase}
  \defaultfontfeatures[\rmfamily]{Ligatures=TeX,Scale=1}
\fi
% Use upquote if available, for straight quotes in verbatim environments
\IfFileExists{upquote.sty}{\usepackage{upquote}}{}
\IfFileExists{microtype.sty}{% use microtype if available
  \usepackage[]{microtype}
  \UseMicrotypeSet[protrusion]{basicmath} % disable protrusion for tt fonts
}{}
\makeatletter
\@ifundefined{KOMAClassName}{% if non-KOMA class
  \IfFileExists{parskip.sty}{%
    \usepackage{parskip}
  }{% else
    \setlength{\parindent}{0pt}
    \setlength{\parskip}{6pt plus 2pt minus 1pt}}
}{% if KOMA class
  \KOMAoptions{parskip=half}}
\makeatother
\usepackage{xcolor}
\usepackage[margin=1in]{geometry}
\usepackage{color}
\usepackage{fancyvrb}
\newcommand{\VerbBar}{|}
\newcommand{\VERB}{\Verb[commandchars=\\\{\}]}
\DefineVerbatimEnvironment{Highlighting}{Verbatim}{commandchars=\\\{\}}
% Add ',fontsize=\small' for more characters per line
\usepackage{framed}
\definecolor{shadecolor}{RGB}{248,248,248}
\newenvironment{Shaded}{\begin{snugshade}}{\end{snugshade}}
\newcommand{\AlertTok}[1]{\textcolor[rgb]{0.94,0.16,0.16}{#1}}
\newcommand{\AnnotationTok}[1]{\textcolor[rgb]{0.56,0.35,0.01}{\textbf{\textit{#1}}}}
\newcommand{\AttributeTok}[1]{\textcolor[rgb]{0.77,0.63,0.00}{#1}}
\newcommand{\BaseNTok}[1]{\textcolor[rgb]{0.00,0.00,0.81}{#1}}
\newcommand{\BuiltInTok}[1]{#1}
\newcommand{\CharTok}[1]{\textcolor[rgb]{0.31,0.60,0.02}{#1}}
\newcommand{\CommentTok}[1]{\textcolor[rgb]{0.56,0.35,0.01}{\textit{#1}}}
\newcommand{\CommentVarTok}[1]{\textcolor[rgb]{0.56,0.35,0.01}{\textbf{\textit{#1}}}}
\newcommand{\ConstantTok}[1]{\textcolor[rgb]{0.00,0.00,0.00}{#1}}
\newcommand{\ControlFlowTok}[1]{\textcolor[rgb]{0.13,0.29,0.53}{\textbf{#1}}}
\newcommand{\DataTypeTok}[1]{\textcolor[rgb]{0.13,0.29,0.53}{#1}}
\newcommand{\DecValTok}[1]{\textcolor[rgb]{0.00,0.00,0.81}{#1}}
\newcommand{\DocumentationTok}[1]{\textcolor[rgb]{0.56,0.35,0.01}{\textbf{\textit{#1}}}}
\newcommand{\ErrorTok}[1]{\textcolor[rgb]{0.64,0.00,0.00}{\textbf{#1}}}
\newcommand{\ExtensionTok}[1]{#1}
\newcommand{\FloatTok}[1]{\textcolor[rgb]{0.00,0.00,0.81}{#1}}
\newcommand{\FunctionTok}[1]{\textcolor[rgb]{0.00,0.00,0.00}{#1}}
\newcommand{\ImportTok}[1]{#1}
\newcommand{\InformationTok}[1]{\textcolor[rgb]{0.56,0.35,0.01}{\textbf{\textit{#1}}}}
\newcommand{\KeywordTok}[1]{\textcolor[rgb]{0.13,0.29,0.53}{\textbf{#1}}}
\newcommand{\NormalTok}[1]{#1}
\newcommand{\OperatorTok}[1]{\textcolor[rgb]{0.81,0.36,0.00}{\textbf{#1}}}
\newcommand{\OtherTok}[1]{\textcolor[rgb]{0.56,0.35,0.01}{#1}}
\newcommand{\PreprocessorTok}[1]{\textcolor[rgb]{0.56,0.35,0.01}{\textit{#1}}}
\newcommand{\RegionMarkerTok}[1]{#1}
\newcommand{\SpecialCharTok}[1]{\textcolor[rgb]{0.00,0.00,0.00}{#1}}
\newcommand{\SpecialStringTok}[1]{\textcolor[rgb]{0.31,0.60,0.02}{#1}}
\newcommand{\StringTok}[1]{\textcolor[rgb]{0.31,0.60,0.02}{#1}}
\newcommand{\VariableTok}[1]{\textcolor[rgb]{0.00,0.00,0.00}{#1}}
\newcommand{\VerbatimStringTok}[1]{\textcolor[rgb]{0.31,0.60,0.02}{#1}}
\newcommand{\WarningTok}[1]{\textcolor[rgb]{0.56,0.35,0.01}{\textbf{\textit{#1}}}}
\usepackage{graphicx}
\makeatletter
\def\maxwidth{\ifdim\Gin@nat@width>\linewidth\linewidth\else\Gin@nat@width\fi}
\def\maxheight{\ifdim\Gin@nat@height>\textheight\textheight\else\Gin@nat@height\fi}
\makeatother
% Scale images if necessary, so that they will not overflow the page
% margins by default, and it is still possible to overwrite the defaults
% using explicit options in \includegraphics[width, height, ...]{}
\setkeys{Gin}{width=\maxwidth,height=\maxheight,keepaspectratio}
% Set default figure placement to htbp
\makeatletter
\def\fps@figure{htbp}
\makeatother
\setlength{\emergencystretch}{3em} % prevent overfull lines
\providecommand{\tightlist}{%
  \setlength{\itemsep}{0pt}\setlength{\parskip}{0pt}}
\setcounter{secnumdepth}{-\maxdimen} % remove section numbering
\ifLuaTeX
  \usepackage{selnolig}  % disable illegal ligatures
\fi
\IfFileExists{bookmark.sty}{\usepackage{bookmark}}{\usepackage{hyperref}}
\IfFileExists{xurl.sty}{\usepackage{xurl}}{} % add URL line breaks if available
\urlstyle{same} % disable monospaced font for URLs
\hypersetup{
  pdftitle={Taxi no supervisado},
  pdfauthor={Daniel Arellano Martínez, Bruno González Arenas, Carlos González Arenas, Víctor Manuel Vázquez García},
  hidelinks,
  pdfcreator={LaTeX via pandoc}}

\title{Taxi no supervisado}
\author{Daniel Arellano Martínez, Bruno González Arenas, Carlos González
Arenas, Víctor Manuel Vázquez García}
\date{}

\begin{document}
\maketitle

\hypertarget{introducciuxf3n}{%
\section{Introducción}\label{introducciuxf3n}}

Los taxis de New York son una forma común de transporte en la ciudad.
Están disponibles en toda la ciudad y pueden ser reconocidos por su
característica pintura amarilla. Los conductores de taxis de Nueva York
deben tener una licencia especial y una tarjeta de identificación, y los
vehículos deben pasar inspecciones regulares para asegurar la seguridad
de los pasajeros. Los pasajeros pueden solicitar un taxi en la calle, a
través de una aplicación móvil o en una parada de taxi designada. Los
taxis también pueden ser contratados para viajes más largos fuera de la
ciudad.

A continuación, se va a analizar un dataset sobre los trayectos de taxis
que incluirá los siguientes datos:

\begin{itemize}
\item
  \emph{id}: identifica a cada viaje
\item
  \emph{vendor\_id}: identifica al proveedor que ofrece el servicio
\item
  \emph{pickup\_datetime}: fecha y hora de recogida
\item
  \emph{dropoff\_datetime}: fecha y hora de llegada
\item
  \emph{passenger\_count}: número de pasajeros (incluyendo el conductor)
\item
  \emph{pickup\_longitude}: longitud donde el pasajero es recogido
\item
  \emph{pickup\_latitude}: latitud donde el pasajero es recogido
\item
  \emph{dropoff\_longitude}: longitud donde el pasajero termina el
  trayecto
\item
  \emph{dropoff\_latitude}: latitud donde el pasajero termina el
  trayecto
\item
  \emph{store\_and\_fwd\_flag}: indica si ha perdido la conexión con el
  servidor y se ha almacenado localmente
\item
  \emph{trip\_duration}: duración del viaje en segundos
\end{itemize}

Este análisis del dataset de taxis de Nueva York se centrará en examinar
las tendencias y patrones en el uso de los taxis en la ciudad. A través
del análisis de datos como la ubicación y el tiempo de viaje, se espera
obtener una mejor comprensión de cómo se utilizan los taxis en Nueva
York y cómo pueden ser mejorados y optimizados para satisfacer las
necesidades de los pasajeros y conductores.

\begin{Shaded}
\begin{Highlighting}[]
\CommentTok{\# importación de paquetes}
\FunctionTok{suppressPackageStartupMessages}\NormalTok{(}\FunctionTok{library}\NormalTok{(dplyr))}
\FunctionTok{library}\NormalTok{(scales)}
\FunctionTok{library}\NormalTok{(geosphere)}
\FunctionTok{library}\NormalTok{(pryr)}
\FunctionTok{library}\NormalTok{(dendextend)}
\end{Highlighting}
\end{Shaded}

\begin{verbatim}
## 
## ---------------------
## Welcome to dendextend version 1.16.0
## Type citation('dendextend') for how to cite the package.
## 
## Type browseVignettes(package = 'dendextend') for the package vignette.
## The github page is: https://github.com/talgalili/dendextend/
## 
## Suggestions and bug-reports can be submitted at: https://github.com/talgalili/dendextend/issues
## You may ask questions at stackoverflow, use the r and dendextend tags: 
##   https://stackoverflow.com/questions/tagged/dendextend
## 
##  To suppress this message use:  suppressPackageStartupMessages(library(dendextend))
## ---------------------
\end{verbatim}

\begin{verbatim}
## 
## Attaching package: 'dendextend'
\end{verbatim}

\begin{verbatim}
## The following object is masked from 'package:stats':
## 
##     cutree
\end{verbatim}

\begin{Shaded}
\begin{Highlighting}[]
\FunctionTok{library}\NormalTok{(ggplot2)}
\FunctionTok{library}\NormalTok{(NbClust)}
\FunctionTok{library}\NormalTok{(factoextra)}
\end{Highlighting}
\end{Shaded}

\begin{verbatim}
## Welcome! Want to learn more? See two factoextra-related books at https://goo.gl/ve3WBa
\end{verbatim}

\begin{Shaded}
\begin{Highlighting}[]
\FunctionTok{library}\NormalTok{(dbscan)}
\end{Highlighting}
\end{Shaded}

\begin{verbatim}
## 
## Attaching package: 'dbscan'
\end{verbatim}

\begin{verbatim}
## The following object is masked from 'package:stats':
## 
##     as.dendrogram
\end{verbatim}

\begin{Shaded}
\begin{Highlighting}[]
\FunctionTok{library}\NormalTok{(caret)}
\end{Highlighting}
\end{Shaded}

\begin{verbatim}
## Loading required package: lattice
\end{verbatim}

\begin{Shaded}
\begin{Highlighting}[]
\FunctionTok{library}\NormalTok{(tidyverse)}
\end{Highlighting}
\end{Shaded}

\begin{verbatim}
## -- Attaching packages --------------------------------------- tidyverse 1.3.2 --
\end{verbatim}

\begin{verbatim}
## v tibble  3.1.8     v purrr   1.0.0
## v tidyr   1.2.1     v stringr 1.5.0
## v readr   2.1.3     v forcats 0.5.2
## -- Conflicts ------------------------------------------ tidyverse_conflicts() --
## x readr::col_factor() masks scales::col_factor()
## x purrr::compose()    masks pryr::compose()
## x purrr::discard()    masks scales::discard()
## x dplyr::filter()     masks stats::filter()
## x dplyr::lag()        masks stats::lag()
## x purrr::lift()       masks caret::lift()
## x purrr::partial()    masks pryr::partial()
\end{verbatim}

\begin{Shaded}
\begin{Highlighting}[]
\FunctionTok{library}\NormalTok{(lubridate)}
\end{Highlighting}
\end{Shaded}

\begin{verbatim}
## Loading required package: timechange
## 
## Attaching package: 'lubridate'
## 
## The following objects are masked from 'package:base':
## 
##     date, intersect, setdiff, union
\end{verbatim}

\begin{Shaded}
\begin{Highlighting}[]
\FunctionTok{library}\NormalTok{(gridExtra)}
\end{Highlighting}
\end{Shaded}

\begin{verbatim}
## 
## Attaching package: 'gridExtra'
## 
## The following object is masked from 'package:dplyr':
## 
##     combine
\end{verbatim}

\begin{Shaded}
\begin{Highlighting}[]
\FunctionTok{library}\NormalTok{(}\StringTok{"xgboost"}\NormalTok{)}
\end{Highlighting}
\end{Shaded}

\begin{verbatim}
## 
## Attaching package: 'xgboost'
## 
## The following object is masked from 'package:dplyr':
## 
##     slice
\end{verbatim}

\hypertarget{obtenciuxf3n-de-datos}{%
\subsection{Obtención de datos}\label{obtenciuxf3n-de-datos}}

Este análisis del dataset de taxis de Nueva York se basa en datos
extraídos de la plataforma
\href{https://www.kaggle.com/competitions/nyc-taxi-trip-duration/data}{Kaggle}
en la sección de competiciones. Estos datos han sido descargados y
tratados de forma local. Para ello se almacenan en la variable/dataset
de R ``train''.

\begin{Shaded}
\begin{Highlighting}[]
\CommentTok{\# dataset de entrenamiento}
\NormalTok{train }\OtherTok{\textless{}{-}} \FunctionTok{read.csv}\NormalTok{(}\StringTok{"datos/train.csv"}\NormalTok{)}

\CommentTok{\# representa el dataset de entrenamiento}
\FunctionTok{head}\NormalTok{(train)}
\end{Highlighting}
\end{Shaded}

\begin{verbatim}
##          id vendor_id     pickup_datetime    dropoff_datetime passenger_count
## 1 id2875421         2 2016-03-14 17:24:55 2016-03-14 17:32:30               1
## 2 id2377394         1 2016-06-12 00:43:35 2016-06-12 00:54:38               1
## 3 id3858529         2 2016-01-19 11:35:24 2016-01-19 12:10:48               1
## 4 id3504673         2 2016-04-06 19:32:31 2016-04-06 19:39:40               1
## 5 id2181028         2 2016-03-26 13:30:55 2016-03-26 13:38:10               1
## 6 id0801584         2 2016-01-30 22:01:40 2016-01-30 22:09:03               6
##   pickup_longitude pickup_latitude dropoff_longitude dropoff_latitude
## 1        -73.98215        40.76794         -73.96463         40.76560
## 2        -73.98042        40.73856         -73.99948         40.73115
## 3        -73.97903        40.76394         -74.00533         40.71009
## 4        -74.01004        40.71997         -74.01227         40.70672
## 5        -73.97305        40.79321         -73.97292         40.78252
## 6        -73.98286        40.74220         -73.99208         40.74918
##   store_and_fwd_flag trip_duration
## 1                  N           455
## 2                  N           663
## 3                  N          2124
## 4                  N           429
## 5                  N           435
## 6                  N           443
\end{verbatim}

\begin{Shaded}
\begin{Highlighting}[]
\CommentTok{\# características básicas del dataset}
\FunctionTok{summary}\NormalTok{(train)}
\end{Highlighting}
\end{Shaded}

\begin{verbatim}
##       id              vendor_id     pickup_datetime    dropoff_datetime  
##  Length:1458644     Min.   :1.000   Length:1458644     Length:1458644    
##  Class :character   1st Qu.:1.000   Class :character   Class :character  
##  Mode  :character   Median :2.000   Mode  :character   Mode  :character  
##                     Mean   :1.535                                        
##                     3rd Qu.:2.000                                        
##                     Max.   :2.000                                        
##  passenger_count pickup_longitude  pickup_latitude dropoff_longitude
##  Min.   :0.000   Min.   :-121.93   Min.   :34.36   Min.   :-121.93  
##  1st Qu.:1.000   1st Qu.: -73.99   1st Qu.:40.74   1st Qu.: -73.99  
##  Median :1.000   Median : -73.98   Median :40.75   Median : -73.98  
##  Mean   :1.665   Mean   : -73.97   Mean   :40.75   Mean   : -73.97  
##  3rd Qu.:2.000   3rd Qu.: -73.97   3rd Qu.:40.77   3rd Qu.: -73.96  
##  Max.   :9.000   Max.   : -61.34   Max.   :51.88   Max.   : -61.34  
##  dropoff_latitude store_and_fwd_flag trip_duration    
##  Min.   :32.18    Length:1458644     Min.   :      1  
##  1st Qu.:40.74    Class :character   1st Qu.:    397  
##  Median :40.75    Mode  :character   Median :    662  
##  Mean   :40.75                       Mean   :    959  
##  3rd Qu.:40.77                       3rd Qu.:   1075  
##  Max.   :43.92                       Max.   :3526282
\end{verbatim}

\hypertarget{supervisado}{%
\section{Supervisado}\label{supervisado}}

\hypertarget{visualizaciuxf3n}{%
\subsection{Visualización}\label{visualizaciuxf3n}}

Empezaremos visualizando cada una de las variables presentes en nuestro
dataset, posteriormente realizaremos un preprocesamiento.

Comenzaremos con vendor\_id, esta variable indica el conductor que
realizó el trayecto.

Se puede observar que el taxista 2 dispone de más trayectos que el 1,
pero la diferencia es insignificado en comparación a la cantidad de
datos existentes.

\begin{Shaded}
\begin{Highlighting}[]
\NormalTok{train }\SpecialCharTok{\%\textgreater{}\%}
  \FunctionTok{mutate}\NormalTok{(}\AttributeTok{vendor\_id =} \FunctionTok{factor}\NormalTok{(vendor\_id)) }\SpecialCharTok{\%\textgreater{}\%}
  \FunctionTok{ggplot}\NormalTok{(}\FunctionTok{aes}\NormalTok{(vendor\_id, }\AttributeTok{fill =}\NormalTok{ vendor\_id)) }\SpecialCharTok{+}
  \FunctionTok{geom\_bar}\NormalTok{() }\SpecialCharTok{+}
  \FunctionTok{scale\_fill\_manual}\NormalTok{(}\AttributeTok{values=}\FunctionTok{c}\NormalTok{(}\StringTok{"red"}\NormalTok{, }\StringTok{"darkblue"}\NormalTok{)) }\SpecialCharTok{+}
  \FunctionTok{labs}\NormalTok{(}\AttributeTok{x =} \StringTok{"Taxista"}\NormalTok{, }\AttributeTok{y =} \StringTok{"Cantidad de trayectos realizados"}\NormalTok{)}
\end{Highlighting}
\end{Shaded}

\includegraphics{taxi_trip_fid_files/figure-latex/unnamed-chunk-4-1.pdf}

Continuamos con passenger\_count, que indica la cantidad de personas que
montaron en el taxi.

Se puede observamos que la mayoría de trayectos unicamente transportaban
una persona.

\begin{Shaded}
\begin{Highlighting}[]
\NormalTok{train }\SpecialCharTok{\%\textgreater{}\%}
  \FunctionTok{ggplot}\NormalTok{(}\FunctionTok{aes}\NormalTok{(passenger\_count)) }\SpecialCharTok{+}
  \FunctionTok{geom\_bar}\NormalTok{(}\AttributeTok{fill =} \StringTok{"\#FF6666"}\NormalTok{) }\SpecialCharTok{+}
  \FunctionTok{labs}\NormalTok{(}\AttributeTok{x =} \StringTok{"Pasajeros"}\NormalTok{, }\AttributeTok{y =} \StringTok{"Cantidad de trayectos realizados"}\NormalTok{)}
\end{Highlighting}
\end{Shaded}

\includegraphics{taxi_trip_fid_files/figure-latex/unnamed-chunk-5-1.pdf}

Ahora analizaremos la variable store\_and\_fwd\_flag.

La mayoría de los registros disponen de valor N, por tanto lo más
probable es que esta columna sea eliminada ya que no aporta información
al problema.

\begin{Shaded}
\begin{Highlighting}[]
\NormalTok{train }\SpecialCharTok{\%\textgreater{}\%} 
    \FunctionTok{ggplot}\NormalTok{(}\FunctionTok{aes}\NormalTok{(store\_and\_fwd\_flag)) }\SpecialCharTok{+}
    \FunctionTok{geom\_bar}\NormalTok{(}\AttributeTok{fill =} \StringTok{"\#FF6666"}\NormalTok{) }\SpecialCharTok{+}
    \FunctionTok{labs}\NormalTok{(}\AttributeTok{x =} \StringTok{"Store and FWD Flash"}\NormalTok{, }\AttributeTok{y =} \StringTok{"Cantidad de trayectos realizados"}\NormalTok{)}
\end{Highlighting}
\end{Shaded}

\includegraphics{taxi_trip_fid_files/figure-latex/unnamed-chunk-6-1.pdf}

En cuanto a las fechas y horas disponibles en nuestro dataset, vamos a
trabajar solo sobre pickup\_time ya que es la que aporta mayor
información al problema.

Vamos a observaremos la distribución de los trayectos respecto a los
meses, las horas y los días de la semana.

Nos podemos dar cuenta que nuestro dataset unicamente contiene viajes
realizados entre enero y junio.

Con estas tres gráficas se puede observar como en las horas de madrugada
la cantidad de viajes baja, y en fin de semana el número sube.

\begin{Shaded}
\begin{Highlighting}[]
\NormalTok{count\_month }\OtherTok{\textless{}{-}}\NormalTok{ train }\SpecialCharTok{\%\textgreater{}\%}
  \FunctionTok{mutate}\NormalTok{(}\AttributeTok{month\_pick =} \FunctionTok{month}\NormalTok{(pickup\_datetime)) }\SpecialCharTok{\%\textgreater{}\%}
  \FunctionTok{group\_by}\NormalTok{(month\_pick) }\SpecialCharTok{\%\textgreater{}\%}
  \FunctionTok{count}\NormalTok{() }\SpecialCharTok{\%\textgreater{}\%}
  \FunctionTok{ggplot}\NormalTok{(}\FunctionTok{aes}\NormalTok{(month\_pick, n)) }\SpecialCharTok{+}
  \FunctionTok{geom\_line}\NormalTok{(}\AttributeTok{linewidth =} \FloatTok{1.5}\NormalTok{, }\AttributeTok{color =} \StringTok{"\#FF6666"}\NormalTok{) }\SpecialCharTok{+}
  \FunctionTok{geom\_point}\NormalTok{(}\AttributeTok{linewidth =} \DecValTok{3}\NormalTok{) }\SpecialCharTok{+} 
  \FunctionTok{labs}\NormalTok{(}\AttributeTok{x =} \StringTok{"Meses"}\NormalTok{, }\AttributeTok{y =} \StringTok{"Cantidad de trayectos realizados"}\NormalTok{)}
\end{Highlighting}
\end{Shaded}

\begin{verbatim}
## Warning in geom_point(linewidth = 3): Ignoring unknown parameters: `linewidth`
\end{verbatim}

\begin{Shaded}
\begin{Highlighting}[]
\NormalTok{count\_hour }\OtherTok{\textless{}{-}}\NormalTok{ train }\SpecialCharTok{\%\textgreater{}\%}
  \FunctionTok{mutate}\NormalTok{(}\AttributeTok{hour\_pick =} \FunctionTok{hour}\NormalTok{(pickup\_datetime)) }\SpecialCharTok{\%\textgreater{}\%}
  \FunctionTok{group\_by}\NormalTok{(hour\_pick) }\SpecialCharTok{\%\textgreater{}\%}
  \FunctionTok{count}\NormalTok{() }\SpecialCharTok{\%\textgreater{}\%}
  \FunctionTok{ggplot}\NormalTok{(}\FunctionTok{aes}\NormalTok{(hour\_pick, n)) }\SpecialCharTok{+}
  \FunctionTok{geom\_line}\NormalTok{(}\AttributeTok{linewidth =} \FloatTok{1.5}\NormalTok{, }\AttributeTok{color =} \StringTok{"blue"}\NormalTok{) }\SpecialCharTok{+}
  \FunctionTok{geom\_point}\NormalTok{(}\AttributeTok{linewidth =} \DecValTok{3}\NormalTok{) }\SpecialCharTok{+} 
  \FunctionTok{labs}\NormalTok{(}\AttributeTok{x =} \StringTok{"Horas"}\NormalTok{, }\AttributeTok{y =} \StringTok{"Cantidad de trayectos realizados"}\NormalTok{)}
\end{Highlighting}
\end{Shaded}

\begin{verbatim}
## Warning in geom_point(linewidth = 3): Ignoring unknown parameters: `linewidth`
\end{verbatim}

\begin{Shaded}
\begin{Highlighting}[]
\NormalTok{count\_wday }\OtherTok{\textless{}{-}}\NormalTok{ train }\SpecialCharTok{\%\textgreater{}\%}
  \FunctionTok{mutate}\NormalTok{(}\AttributeTok{week\_day =} \FunctionTok{wday}\NormalTok{(pickup\_datetime, }\AttributeTok{week\_start =} \DecValTok{1}\NormalTok{)) }\SpecialCharTok{\%\textgreater{}\%}
  \FunctionTok{group\_by}\NormalTok{(week\_day) }\SpecialCharTok{\%\textgreater{}\%}
  \FunctionTok{count}\NormalTok{() }\SpecialCharTok{\%\textgreater{}\%}
  \FunctionTok{ggplot}\NormalTok{(}\FunctionTok{aes}\NormalTok{(week\_day, n)) }\SpecialCharTok{+}
  \FunctionTok{geom\_line}\NormalTok{(}\AttributeTok{linewidth =} \FloatTok{1.5}\NormalTok{, }\AttributeTok{color =} \StringTok{"yellow"}\NormalTok{) }\SpecialCharTok{+}
  \FunctionTok{geom\_point}\NormalTok{(}\AttributeTok{linewidth =} \DecValTok{3}\NormalTok{) }\SpecialCharTok{+} 
  \FunctionTok{labs}\NormalTok{(}\AttributeTok{x =} \StringTok{"Días de la semana"}\NormalTok{, }\AttributeTok{y =} \StringTok{"Cantidad de trayectos realizados"}\NormalTok{)}
\end{Highlighting}
\end{Shaded}

\begin{verbatim}
## Warning in geom_point(linewidth = 3): Ignoring unknown parameters: `linewidth`
\end{verbatim}

\begin{Shaded}
\begin{Highlighting}[]
\FunctionTok{grid.arrange}\NormalTok{(count\_month, count\_hour, count\_wday, }\AttributeTok{nrow =} \DecValTok{2}\NormalTok{, }\AttributeTok{ncol =} \DecValTok{2}\NormalTok{)}
\end{Highlighting}
\end{Shaded}

\includegraphics{taxi_trip_fid_files/figure-latex/unnamed-chunk-7-1.pdf}

\begin{Shaded}
\begin{Highlighting}[]
\FunctionTok{rm}\NormalTok{(count\_month, count\_hour, count\_wday)}
\end{Highlighting}
\end{Shaded}

Continuando con la columna pickup\_datetime, vamos a ver como evoluciona
la cantidad de trayectos cada mes según las horas y los días de la
semana.

\begin{Shaded}
\begin{Highlighting}[]
\NormalTok{month\_hour }\OtherTok{\textless{}{-}}\NormalTok{ train }\SpecialCharTok{\%\textgreater{}\%}
  \FunctionTok{mutate}\NormalTok{(}\AttributeTok{hour\_pick =} \FunctionTok{hour}\NormalTok{(pickup\_datetime),}
         \AttributeTok{month\_pick =} \FunctionTok{factor}\NormalTok{(}\FunctionTok{month}\NormalTok{(pickup\_datetime, }\AttributeTok{label =} \ConstantTok{TRUE}\NormalTok{))) }\SpecialCharTok{\%\textgreater{}\%}
  \FunctionTok{group\_by}\NormalTok{(hour\_pick, month\_pick) }\SpecialCharTok{\%\textgreater{}\%}
  \FunctionTok{count}\NormalTok{() }\SpecialCharTok{\%\textgreater{}\%}
  \FunctionTok{ggplot}\NormalTok{(}\FunctionTok{aes}\NormalTok{(hour\_pick, n, }\AttributeTok{color =}\NormalTok{ month\_pick)) }\SpecialCharTok{+}
  \FunctionTok{geom\_line}\NormalTok{(}\AttributeTok{size =} \FloatTok{1.5}\NormalTok{) }\SpecialCharTok{+}
  \FunctionTok{labs}\NormalTok{(}\AttributeTok{x =} \StringTok{"Horas"}\NormalTok{, }\AttributeTok{y =} \StringTok{"Cantidad de trayectos realizados"}\NormalTok{)}
\end{Highlighting}
\end{Shaded}

\begin{verbatim}
## Warning: Using `size` aesthetic for lines was deprecated in ggplot2 3.4.0.
## i Please use `linewidth` instead.
\end{verbatim}

\begin{Shaded}
\begin{Highlighting}[]
\NormalTok{hour\_wday }\OtherTok{\textless{}{-}}\NormalTok{ train }\SpecialCharTok{\%\textgreater{}\%}
  \FunctionTok{mutate}\NormalTok{(}\AttributeTok{hour\_pick =} \FunctionTok{hour}\NormalTok{(pickup\_datetime),}
         \AttributeTok{week\_day =} \FunctionTok{factor}\NormalTok{(}\FunctionTok{wday}\NormalTok{(pickup\_datetime, }\AttributeTok{label =} \ConstantTok{TRUE}\NormalTok{, }\AttributeTok{week\_start =} \DecValTok{1}\NormalTok{))) }\SpecialCharTok{\%\textgreater{}\%}
  \FunctionTok{group\_by}\NormalTok{(hour\_pick, week\_day) }\SpecialCharTok{\%\textgreater{}\%}
  \FunctionTok{count}\NormalTok{() }\SpecialCharTok{\%\textgreater{}\%}
  \FunctionTok{ggplot}\NormalTok{(}\FunctionTok{aes}\NormalTok{(hour\_pick, n, }\AttributeTok{color =}\NormalTok{ week\_day)) }\SpecialCharTok{+}
  \FunctionTok{geom\_line}\NormalTok{(}\AttributeTok{size =} \FloatTok{1.5}\NormalTok{) }\SpecialCharTok{+}
  \FunctionTok{labs}\NormalTok{(}\AttributeTok{x =} \StringTok{"Día de la semana"}\NormalTok{, }\AttributeTok{y =} \StringTok{"Cantidad de trayectos realizados"}\NormalTok{)}

\FunctionTok{grid.arrange}\NormalTok{(month\_hour, hour\_wday, }\AttributeTok{nrow =} \DecValTok{2}\NormalTok{)}
\end{Highlighting}
\end{Shaded}

\includegraphics{taxi_trip_fid_files/figure-latex/unnamed-chunk-8-1.pdf}

\begin{Shaded}
\begin{Highlighting}[]
\FunctionTok{rm}\NormalTok{(month\_hour, hour\_wday)}
\end{Highlighting}
\end{Shaded}

Un campo importante a la hora de visualizar es la distancia que existe
entre el punto de recogida y destino.

Vamos a crear dos nuevas columnas que contengan esta información.
Crearemos una con la distancia euclídea y otra con la manhattan, y
compararemos los resultados

\begin{Shaded}
\begin{Highlighting}[]
\NormalTok{euclidean }\OtherTok{\textless{}{-}} \ControlFlowTok{function}\NormalTok{(x1, x2, y1, y2) }\FunctionTok{sqrt}\NormalTok{((x1 }\SpecialCharTok{{-}}\NormalTok{ y1)}\SpecialCharTok{\^{}}\DecValTok{2} \SpecialCharTok{+}\NormalTok{ (x2 }\SpecialCharTok{{-}}\NormalTok{ y2)}\SpecialCharTok{\^{}}\DecValTok{2}\NormalTok{)}
\NormalTok{manhattan }\OtherTok{\textless{}{-}} \ControlFlowTok{function}\NormalTok{(x1, x2, y1, y2)\{}
\NormalTok{    dist }\OtherTok{\textless{}{-}} \FunctionTok{abs}\NormalTok{(x1}\SpecialCharTok{{-}}\NormalTok{y1) }\SpecialCharTok{+} \FunctionTok{abs}\NormalTok{(x2}\SpecialCharTok{{-}}\NormalTok{y2)}
    \FunctionTok{return}\NormalTok{(dist)}
\NormalTok{\}}

\NormalTok{train }\OtherTok{\textless{}{-}}\NormalTok{ train }\SpecialCharTok{\%\textgreater{}\%} \FunctionTok{mutate}\NormalTok{(}\AttributeTok{distance\_euclidean =} \FunctionTok{euclidean}\NormalTok{(pickup\_latitude, pickup\_longitude, dropoff\_latitude, dropoff\_longitude),}
                                          \AttributeTok{distance\_manhattan =} \FunctionTok{manhattan}\NormalTok{(pickup\_latitude, pickup\_longitude, dropoff\_latitude, dropoff\_longitude))}


\FunctionTok{head}\NormalTok{(train, }\DecValTok{5}\NormalTok{)}
\end{Highlighting}
\end{Shaded}

\begin{verbatim}
##          id vendor_id     pickup_datetime    dropoff_datetime passenger_count
## 1 id2875421         2 2016-03-14 17:24:55 2016-03-14 17:32:30               1
## 2 id2377394         1 2016-06-12 00:43:35 2016-06-12 00:54:38               1
## 3 id3858529         2 2016-01-19 11:35:24 2016-01-19 12:10:48               1
## 4 id3504673         2 2016-04-06 19:32:31 2016-04-06 19:39:40               1
## 5 id2181028         2 2016-03-26 13:30:55 2016-03-26 13:38:10               1
##   pickup_longitude pickup_latitude dropoff_longitude dropoff_latitude
## 1        -73.98215        40.76794         -73.96463         40.76560
## 2        -73.98042        40.73856         -73.99948         40.73115
## 3        -73.97903        40.76394         -74.00533         40.71009
## 4        -74.01004        40.71997         -74.01227         40.70672
## 5        -73.97305        40.79321         -73.97292         40.78252
##   store_and_fwd_flag trip_duration distance_euclidean distance_manhattan
## 1                  N           455         0.01767954         0.01985931
## 2                  N           663         0.02045590         0.02647781
## 3                  N          2124         0.05993380         0.08015823
## 4                  N           429         0.01343821         0.01548004
## 5                  N           435         0.01068957         0.01081848
\end{verbatim}

Si visualizamos ambas podemos ver que son muy similares, algo de
esperar.

Podemos apreciar que la mayoría de los trayectos tienen una distancia
similar, aunque existen 2 puntos donde esta tendencia cambia.

\begin{Shaded}
\begin{Highlighting}[]
\NormalTok{distancia1 }\OtherTok{\textless{}{-}}\NormalTok{ train }\SpecialCharTok{\%\textgreater{}\%}
  \FunctionTok{ggplot}\NormalTok{(}\FunctionTok{aes}\NormalTok{(distance\_euclidean)) }\SpecialCharTok{+}
  \FunctionTok{geom\_density}\NormalTok{(}\FunctionTok{aes}\NormalTok{(}\AttributeTok{y=}\FunctionTok{after\_stat}\NormalTok{(count)), }\AttributeTok{color=}\StringTok{"darkblue"}\NormalTok{, }\AttributeTok{fill=}\StringTok{"darkblue"}\NormalTok{) }\SpecialCharTok{+}
  \FunctionTok{scale\_x\_log10}\NormalTok{() }\SpecialCharTok{+}
  \FunctionTok{labs}\NormalTok{(}\AttributeTok{x =} \StringTok{"Distancia Euclídea"}\NormalTok{, }\AttributeTok{y =} \StringTok{"Trayectos realizados"}\NormalTok{)}

\NormalTok{distancia2 }\OtherTok{\textless{}{-}}\NormalTok{ train }\SpecialCharTok{\%\textgreater{}\%}
  \FunctionTok{ggplot}\NormalTok{(}\FunctionTok{aes}\NormalTok{(distance\_manhattan)) }\SpecialCharTok{+}
  \FunctionTok{geom\_density}\NormalTok{(}\FunctionTok{aes}\NormalTok{(}\AttributeTok{y=}\FunctionTok{after\_stat}\NormalTok{(count)), }\AttributeTok{color=}\StringTok{"red"}\NormalTok{, }\AttributeTok{fill=}\StringTok{"red"}\NormalTok{) }\SpecialCharTok{+}
  \FunctionTok{scale\_x\_log10}\NormalTok{() }\SpecialCharTok{+}
  \FunctionTok{labs}\NormalTok{(}\AttributeTok{x =} \StringTok{"Distancia Manhattan"}\NormalTok{, }\AttributeTok{y =} \StringTok{"Trayectos realizados"}\NormalTok{)}

\FunctionTok{grid.arrange}\NormalTok{(distancia1, distancia2, }\AttributeTok{nrow=}\DecValTok{1}\NormalTok{)}
\end{Highlighting}
\end{Shaded}

\begin{verbatim}
## Warning: Transformation introduced infinite values in continuous x-axis
\end{verbatim}

\begin{verbatim}
## Warning: Removed 5897 rows containing non-finite values (`stat_density()`).
\end{verbatim}

\begin{verbatim}
## Warning: Transformation introduced infinite values in continuous x-axis
\end{verbatim}

\begin{verbatim}
## Warning: Removed 5897 rows containing non-finite values (`stat_density()`).
\end{verbatim}

\includegraphics{taxi_trip_fid_files/figure-latex/unnamed-chunk-10-1.pdf}

\begin{Shaded}
\begin{Highlighting}[]
\FunctionTok{rm}\NormalTok{(distancia1, distancia2)}
\end{Highlighting}
\end{Shaded}

Si visualizamos las coordenadas de las sitios de recogida podemos
observer la siguiente gráfica.

\begin{Shaded}
\begin{Highlighting}[]
\NormalTok{train }\SpecialCharTok{\%\textgreater{}\%}
  \FunctionTok{ggplot}\NormalTok{(}\FunctionTok{aes}\NormalTok{(pickup\_longitude, pickup\_latitude)) }\SpecialCharTok{+}
  \FunctionTok{geom\_point}\NormalTok{(}\AttributeTok{color=}\StringTok{\textquotesingle{}red\textquotesingle{}}\NormalTok{) }\SpecialCharTok{+}
  \FunctionTok{xlim}\NormalTok{(}\SpecialCharTok{{-}}\DecValTok{75}\NormalTok{, }\SpecialCharTok{{-}}\FloatTok{72.5}\NormalTok{) }\SpecialCharTok{+}
  \FunctionTok{ylim}\NormalTok{(}\DecValTok{40}\NormalTok{, }\FloatTok{42.5}\NormalTok{) }\SpecialCharTok{+}
  \FunctionTok{labs}\NormalTok{(}\AttributeTok{x =} \StringTok{"Longitud"}\NormalTok{, }\AttributeTok{y =} \StringTok{"Latitud"}\NormalTok{)}
\end{Highlighting}
\end{Shaded}

\begin{verbatim}
## Warning: Removed 31 rows containing missing values (`geom_point()`).
\end{verbatim}

\includegraphics{taxi_trip_fid_files/figure-latex/unnamed-chunk-11-1.pdf}

Para finalizar con este punto vamos a ver información sobre nuestra
columna objetivo, trip\_duration.

Empezaremos viendo la distribución de las duraciones de los trayectos.
Como se puede observar el resultado es muy simular a una campana de
Gauss.

\begin{Shaded}
\begin{Highlighting}[]
\NormalTok{train }\SpecialCharTok{\%\textgreater{}\%}
  \FunctionTok{ggplot}\NormalTok{(}\FunctionTok{aes}\NormalTok{(trip\_duration)) }\SpecialCharTok{+}
  \FunctionTok{geom\_density}\NormalTok{(}\FunctionTok{aes}\NormalTok{(}\AttributeTok{y=}\FunctionTok{after\_stat}\NormalTok{(count)), }\AttributeTok{color=}\StringTok{"darkblue"}\NormalTok{, }\AttributeTok{fill=}\StringTok{"lightblue"}\NormalTok{) }\SpecialCharTok{+}
  \FunctionTok{scale\_x\_log10}\NormalTok{() }\SpecialCharTok{+}
  \FunctionTok{labs}\NormalTok{(}\AttributeTok{x =} \StringTok{"Duración del trayecto"}\NormalTok{, }\AttributeTok{y =} \StringTok{"Trayectos realizados"}\NormalTok{)}
\end{Highlighting}
\end{Shaded}

\includegraphics{taxi_trip_fid_files/figure-latex/unnamed-chunk-12-1.pdf}

Ahora analizaremos la media y la mediana de la duración de los viajes
diferenciando por vendor/taxista.

Se puede observar que la mediana de ambos es muy similar.

\begin{Shaded}
\begin{Highlighting}[]
\NormalTok{dur\_vendor\_1 }\OtherTok{\textless{}{-}}\NormalTok{ train }\SpecialCharTok{\%\textgreater{}\%}
  \FunctionTok{group\_by}\NormalTok{(vendor\_id) }\SpecialCharTok{\%\textgreater{}\%}
  \FunctionTok{summarise}\NormalTok{(}\AttributeTok{mean\_duration =} \FunctionTok{mean}\NormalTok{(trip\_duration)) }\SpecialCharTok{\%\textgreater{}\%}
  \FunctionTok{ggplot}\NormalTok{(}\FunctionTok{aes}\NormalTok{(}\AttributeTok{x=}\NormalTok{vendor\_id, }\AttributeTok{y=}\NormalTok{mean\_duration)) }\SpecialCharTok{+}
  \FunctionTok{geom\_col}\NormalTok{(}\AttributeTok{position=}\StringTok{"dodge"}\NormalTok{, }\AttributeTok{color=}\StringTok{\textquotesingle{}red\textquotesingle{}}\NormalTok{) }\SpecialCharTok{+}
  \FunctionTok{labs}\NormalTok{(}\AttributeTok{x =} \StringTok{"Taxista"}\NormalTok{, }\AttributeTok{y =} \StringTok{"Media"}\NormalTok{)}

\NormalTok{dur\_vendor\_2 }\OtherTok{\textless{}{-}}\NormalTok{ train }\SpecialCharTok{\%\textgreater{}\%}
  \FunctionTok{group\_by}\NormalTok{(vendor\_id) }\SpecialCharTok{\%\textgreater{}\%}
  \FunctionTok{summarise}\NormalTok{(}\AttributeTok{median\_duration =} \FunctionTok{median}\NormalTok{(trip\_duration)) }\SpecialCharTok{\%\textgreater{}\%}
  \FunctionTok{ggplot}\NormalTok{(}\FunctionTok{aes}\NormalTok{(}\AttributeTok{x=}\NormalTok{vendor\_id, }\AttributeTok{y=}\NormalTok{median\_duration)) }\SpecialCharTok{+}
  \FunctionTok{geom\_col}\NormalTok{(}\AttributeTok{position=}\StringTok{"dodge"}\NormalTok{, }\AttributeTok{color=}\StringTok{\textquotesingle{}blue\textquotesingle{}}\NormalTok{) }\SpecialCharTok{+}
  \FunctionTok{labs}\NormalTok{(}\AttributeTok{x =} \StringTok{"Taxista"}\NormalTok{, }\AttributeTok{y =} \StringTok{"Mediana"}\NormalTok{)}

\FunctionTok{grid.arrange}\NormalTok{(dur\_vendor\_1, dur\_vendor\_2, }\AttributeTok{ncol=}\DecValTok{1}\NormalTok{)}
\end{Highlighting}
\end{Shaded}

\includegraphics{taxi_trip_fid_files/figure-latex/unnamed-chunk-13-1.pdf}

\begin{Shaded}
\begin{Highlighting}[]
\FunctionTok{rm}\NormalTok{(dur\_vendor\_1, dur\_vendor\_2)}
\end{Highlighting}
\end{Shaded}

Ahora visualizaremos la mediana de las duraciones de los viajes según el
número de pasajeros.

La primera gráfica nos muestra un número muy alto de duración de los
viajes cuando el número de pasajeros en 0, y muy bajo cuando es 7, 8 y
9.

Si miramos la siguiente tabla podemos ver que justamente esos casos
disponen de muy pocos registros en el dataset y por tanto se pueden
considerar como datos no válidos (posiblemente sean errores).

Si ignoramos esa columna vemos que la duración de los trayectos respecto
a la cantidad de pasajeros es parecido.

\begin{Shaded}
\begin{Highlighting}[]
\NormalTok{train }\SpecialCharTok{\%\textgreater{}\%}
  \FunctionTok{group\_by}\NormalTok{(passenger\_count) }\SpecialCharTok{\%\textgreater{}\%}
  \FunctionTok{summarise}\NormalTok{(}\AttributeTok{mean\_duration =} \FunctionTok{mean}\NormalTok{(trip\_duration)) }\SpecialCharTok{\%\textgreater{}\%}
  \FunctionTok{ggplot}\NormalTok{(}\FunctionTok{aes}\NormalTok{(}\AttributeTok{x=}\NormalTok{passenger\_count, }\AttributeTok{y=}\NormalTok{mean\_duration, }\AttributeTok{fill=}\NormalTok{mean\_duration)) }\SpecialCharTok{+}
  \FunctionTok{geom\_col}\NormalTok{(}\AttributeTok{position=}\StringTok{"dodge"}\NormalTok{) }\SpecialCharTok{+}
  \FunctionTok{labs}\NormalTok{(}\AttributeTok{x =} \StringTok{"Cantidad de pasajeros"}\NormalTok{, }\AttributeTok{y =} \StringTok{"Mediana trip\_duration"}\NormalTok{)}
\end{Highlighting}
\end{Shaded}

\includegraphics{taxi_trip_fid_files/figure-latex/unnamed-chunk-14-1.pdf}

\begin{Shaded}
\begin{Highlighting}[]
\NormalTok{train }\SpecialCharTok{\%\textgreater{}\%}
  \FunctionTok{group\_by}\NormalTok{(passenger\_count) }\SpecialCharTok{\%\textgreater{}\%}
  \FunctionTok{count}\NormalTok{()}
\end{Highlighting}
\end{Shaded}

\begin{verbatim}
## # A tibble: 10 x 2
## # Groups:   passenger_count [10]
##    passenger_count       n
##              <int>   <int>
##  1               0      60
##  2               1 1033540
##  3               2  210318
##  4               3   59896
##  5               4   28404
##  6               5   78088
##  7               6   48333
##  8               7       3
##  9               8       1
## 10               9       1
\end{verbatim}

Continuamos con pickup\_datetime. Vamos a ver como evoluciona la
duración de los viajes según los meses, las horas, y los días de la
semana.

Aunque se observen picos la diferencia entre los valores es mínima.

\begin{Shaded}
\begin{Highlighting}[]
\NormalTok{dur\_month }\OtherTok{\textless{}{-}}\NormalTok{ train }\SpecialCharTok{\%\textgreater{}\%}
  \FunctionTok{mutate}\NormalTok{(}\AttributeTok{month\_pick =} \FunctionTok{month}\NormalTok{(pickup\_datetime)) }\SpecialCharTok{\%\textgreater{}\%}
  \FunctionTok{group\_by}\NormalTok{(month\_pick) }\SpecialCharTok{\%\textgreater{}\%}
  \FunctionTok{summarise}\NormalTok{(}\AttributeTok{mean\_duration =} \FunctionTok{mean}\NormalTok{(trip\_duration)) }\SpecialCharTok{\%\textgreater{}\%}
  \FunctionTok{ggplot}\NormalTok{(}\FunctionTok{aes}\NormalTok{(month\_pick, mean\_duration)) }\SpecialCharTok{+}
  \FunctionTok{geom\_line}\NormalTok{(}\AttributeTok{size =} \FloatTok{1.5}\NormalTok{, }\AttributeTok{color =} \StringTok{"\#FF6666"}\NormalTok{) }\SpecialCharTok{+}
  \FunctionTok{geom\_point}\NormalTok{(}\AttributeTok{size =} \DecValTok{3}\NormalTok{) }\SpecialCharTok{+} 
  \FunctionTok{labs}\NormalTok{(}\AttributeTok{x =} \StringTok{"Meses"}\NormalTok{, }\AttributeTok{y =} \StringTok{"Duración de trayectos realizados"}\NormalTok{)}

\NormalTok{dur\_hour }\OtherTok{\textless{}{-}}\NormalTok{ train }\SpecialCharTok{\%\textgreater{}\%}
  \FunctionTok{mutate}\NormalTok{(}\AttributeTok{hour\_pick =} \FunctionTok{hour}\NormalTok{(pickup\_datetime)) }\SpecialCharTok{\%\textgreater{}\%}
  \FunctionTok{group\_by}\NormalTok{(hour\_pick) }\SpecialCharTok{\%\textgreater{}\%}
  \FunctionTok{summarise}\NormalTok{(}\AttributeTok{mean\_duration =} \FunctionTok{mean}\NormalTok{(trip\_duration)) }\SpecialCharTok{\%\textgreater{}\%}
  \FunctionTok{ggplot}\NormalTok{(}\FunctionTok{aes}\NormalTok{(hour\_pick, mean\_duration)) }\SpecialCharTok{+}
  \FunctionTok{geom\_line}\NormalTok{(}\AttributeTok{size =} \FloatTok{1.5}\NormalTok{, }\AttributeTok{color =} \StringTok{"blue"}\NormalTok{) }\SpecialCharTok{+}
  \FunctionTok{geom\_point}\NormalTok{(}\AttributeTok{size =} \DecValTok{3}\NormalTok{) }\SpecialCharTok{+} 
  \FunctionTok{labs}\NormalTok{(}\AttributeTok{x =} \StringTok{"Horas"}\NormalTok{, }\AttributeTok{y =} \StringTok{"Duración de trayectos realizados"}\NormalTok{)}

\NormalTok{dur\_wday }\OtherTok{\textless{}{-}}\NormalTok{ train }\SpecialCharTok{\%\textgreater{}\%}
  \FunctionTok{mutate}\NormalTok{(}\AttributeTok{week\_day =} \FunctionTok{wday}\NormalTok{(pickup\_datetime, }\AttributeTok{week\_start =} \DecValTok{1}\NormalTok{)) }\SpecialCharTok{\%\textgreater{}\%}
  \FunctionTok{group\_by}\NormalTok{(week\_day) }\SpecialCharTok{\%\textgreater{}\%}
  \FunctionTok{summarise}\NormalTok{(}\AttributeTok{mean\_duration =} \FunctionTok{mean}\NormalTok{(trip\_duration)) }\SpecialCharTok{\%\textgreater{}\%}
  \FunctionTok{ggplot}\NormalTok{(}\FunctionTok{aes}\NormalTok{(week\_day, mean\_duration)) }\SpecialCharTok{+}
  \FunctionTok{geom\_line}\NormalTok{(}\AttributeTok{size =} \FloatTok{1.5}\NormalTok{, }\AttributeTok{color =} \StringTok{"yellow"}\NormalTok{) }\SpecialCharTok{+}
  \FunctionTok{geom\_point}\NormalTok{(}\AttributeTok{size =} \DecValTok{3}\NormalTok{) }\SpecialCharTok{+} 
  \FunctionTok{labs}\NormalTok{(}\AttributeTok{x =} \StringTok{"Días de la semana"}\NormalTok{, }\AttributeTok{y =} \StringTok{"Duración de trayectos realizados"}\NormalTok{)}

\FunctionTok{grid.arrange}\NormalTok{(dur\_month, dur\_hour, dur\_wday, }\AttributeTok{ncol =} \DecValTok{2}\NormalTok{)}
\end{Highlighting}
\end{Shaded}

\includegraphics{taxi_trip_fid_files/figure-latex/unnamed-chunk-15-1.pdf}

\begin{Shaded}
\begin{Highlighting}[]
\FunctionTok{rm}\NormalTok{(dur\_month, dur\_hour, dur\_wday)}
\end{Highlighting}
\end{Shaded}

Para concluir con este apartado vamos a ver la distribución de las
distancias respecto a la duración de los trayectos.

Como se puede observar, ambas distancias generan una gráfica muy
simular.

Por tanto, no podemos decidir a simple vista que distancia mejorará los
resultados de nuestra preducción.

\begin{Shaded}
\begin{Highlighting}[]
\FunctionTok{set.seed}\NormalTok{(}\DecValTok{2}\NormalTok{)}
\NormalTok{dist\_dur\_1 }\OtherTok{\textless{}{-}}\NormalTok{ train }\SpecialCharTok{\%\textgreater{}\%}
  \FunctionTok{sample\_n}\NormalTok{(}\FloatTok{5e4}\NormalTok{) }\SpecialCharTok{\%\textgreater{}\%}
  \FunctionTok{ggplot}\NormalTok{(}\FunctionTok{aes}\NormalTok{(distance\_euclidean, trip\_duration)) }\SpecialCharTok{+}
  \FunctionTok{geom\_point}\NormalTok{(}\AttributeTok{color=}\StringTok{\textquotesingle{}red\textquotesingle{}}\NormalTok{) }\SpecialCharTok{+}
  \FunctionTok{scale\_x\_log10}\NormalTok{() }\SpecialCharTok{+}
  \FunctionTok{scale\_y\_log10}\NormalTok{() }\SpecialCharTok{+}
  \FunctionTok{labs}\NormalTok{(}\AttributeTok{x =} \StringTok{"Distancia Euclídea"}\NormalTok{, }\AttributeTok{y =} \StringTok{"Duración del trayecto"}\NormalTok{)}

\NormalTok{dist\_dur\_2 }\OtherTok{\textless{}{-}}\NormalTok{ train }\SpecialCharTok{\%\textgreater{}\%}
  \FunctionTok{sample\_n}\NormalTok{(}\FloatTok{5e4}\NormalTok{) }\SpecialCharTok{\%\textgreater{}\%}
  \FunctionTok{ggplot}\NormalTok{(}\FunctionTok{aes}\NormalTok{(distance\_manhattan, trip\_duration)) }\SpecialCharTok{+}
  \FunctionTok{geom\_point}\NormalTok{(}\AttributeTok{color=}\StringTok{\textquotesingle{}blue\textquotesingle{}}\NormalTok{) }\SpecialCharTok{+}
  \FunctionTok{scale\_x\_log10}\NormalTok{() }\SpecialCharTok{+}
  \FunctionTok{scale\_y\_log10}\NormalTok{() }\SpecialCharTok{+}
  \FunctionTok{labs}\NormalTok{(}\AttributeTok{x =} \StringTok{"Distancia Manhattan"}\NormalTok{, }\AttributeTok{y =} \StringTok{"Duración del trayecto"}\NormalTok{)}

\FunctionTok{grid.arrange}\NormalTok{(dist\_dur\_1, dist\_dur\_2, }\AttributeTok{ncol=}\DecValTok{2}\NormalTok{)}
\end{Highlighting}
\end{Shaded}

\begin{verbatim}
## Warning: Transformation introduced infinite values in continuous x-axis
## Transformation introduced infinite values in continuous x-axis
\end{verbatim}

\includegraphics{taxi_trip_fid_files/figure-latex/unnamed-chunk-16-1.pdf}

\begin{Shaded}
\begin{Highlighting}[]
\FunctionTok{rm}\NormalTok{(dist\_dur\_1, dist\_dur\_2)}
\end{Highlighting}
\end{Shaded}

\hypertarget{preprocesamiento}{%
\subsection{Preprocesamiento}\label{preprocesamiento}}

Tras haber visualizado los datos podemos hacernos una idea de que datos
debemos modificar/ajustar en nuestro dataset.

Aun así comenzaremos viendo un resumen del dataset.

\begin{Shaded}
\begin{Highlighting}[]
\FunctionTok{summary}\NormalTok{(train)}
\end{Highlighting}
\end{Shaded}

\begin{verbatim}
##       id              vendor_id     pickup_datetime    dropoff_datetime  
##  Length:1458644     Min.   :1.000   Length:1458644     Length:1458644    
##  Class :character   1st Qu.:1.000   Class :character   Class :character  
##  Mode  :character   Median :2.000   Mode  :character   Mode  :character  
##                     Mean   :1.535                                        
##                     3rd Qu.:2.000                                        
##                     Max.   :2.000                                        
##  passenger_count pickup_longitude  pickup_latitude dropoff_longitude
##  Min.   :0.000   Min.   :-121.93   Min.   :34.36   Min.   :-121.93  
##  1st Qu.:1.000   1st Qu.: -73.99   1st Qu.:40.74   1st Qu.: -73.99  
##  Median :1.000   Median : -73.98   Median :40.75   Median : -73.98  
##  Mean   :1.665   Mean   : -73.97   Mean   :40.75   Mean   : -73.97  
##  3rd Qu.:2.000   3rd Qu.: -73.97   3rd Qu.:40.77   3rd Qu.: -73.96  
##  Max.   :9.000   Max.   : -61.34   Max.   :51.88   Max.   : -61.34  
##  dropoff_latitude store_and_fwd_flag trip_duration     distance_euclidean
##  Min.   :32.18    Length:1458644     Min.   :      1   Min.   : 0.00000  
##  1st Qu.:40.74    Class :character   1st Qu.:    397   1st Qu.: 0.01258  
##  Median :40.75    Mode  :character   Median :    662   Median : 0.02122  
##  Mean   :40.75                       Mean   :    959   Mean   : 0.03548  
##  3rd Qu.:40.77                       3rd Qu.:   1075   3rd Qu.: 0.03841  
##  Max.   :43.92                       Max.   :3526282   Max.   :11.19260  
##  distance_manhattan
##  Min.   : 0.00000  
##  1st Qu.: 0.01609  
##  Median : 0.02738  
##  Mean   : 0.04590  
##  3rd Qu.: 0.05048  
##  Max.   :12.90774
\end{verbatim}

Podemos observar que nuestro dataset no dispone de ningún dato nulo.

Procederemos a eliminar la columna id, esta es generada aleatoriamente
por lo que no aporta información.

\begin{Shaded}
\begin{Highlighting}[]
\NormalTok{train }\OtherTok{\textless{}{-}} \FunctionTok{select}\NormalTok{(train, }\SpecialCharTok{{-}}\NormalTok{id)}
\end{Highlighting}
\end{Shaded}

La columna \texttt{vendor\_id} se encuentra en formato númerico y por lo
que hemos visto en el apartado anterior puede ser importante a la hora
de predecir.

La columna \texttt{dropoff\_datetime} no aporta una información
relevante al problema ya que es la suma de \texttt{pickup\_datetime} y
\texttt{trip\_duration}.

Procedemos a borrarla

\begin{Shaded}
\begin{Highlighting}[]
\NormalTok{train }\OtherTok{\textless{}{-}}\NormalTok{ train }\SpecialCharTok{\%\textgreater{}\%} \FunctionTok{select}\NormalTok{(}\SpecialCharTok{{-}}\NormalTok{dropoff\_datetime)}
\end{Highlighting}
\end{Shaded}

Sin embargo, la columna \texttt{pickup\_datetime} si es importante, pero
es necesario que realicemos un procesamiento ya que está en formato
fecha.

Vamos a dividir la columna en 3, mes, hora y día de la semana.

\begin{Shaded}
\begin{Highlighting}[]
\NormalTok{train }\OtherTok{\textless{}{-}}\NormalTok{ train }\SpecialCharTok{\%\textgreater{}\%} \FunctionTok{mutate}\NormalTok{(}\AttributeTok{month =} \FunctionTok{month}\NormalTok{(pickup\_datetime))}
\NormalTok{train }\OtherTok{\textless{}{-}}\NormalTok{ train }\SpecialCharTok{\%\textgreater{}\%} \FunctionTok{mutate}\NormalTok{(}\AttributeTok{hour =} \FunctionTok{hour}\NormalTok{(pickup\_datetime))}
\NormalTok{train }\OtherTok{\textless{}{-}}\NormalTok{ train }\SpecialCharTok{\%\textgreater{}\%} \FunctionTok{mutate}\NormalTok{(}\AttributeTok{week\_day =} \FunctionTok{wday}\NormalTok{(pickup\_datetime, }\AttributeTok{week\_start =} \DecValTok{1}\NormalTok{))}

\NormalTok{train }\OtherTok{\textless{}{-}}\NormalTok{ train }\SpecialCharTok{\%\textgreater{}\%} \FunctionTok{select}\NormalTok{(}\SpecialCharTok{{-}}\NormalTok{pickup\_datetime)}
\end{Highlighting}
\end{Shaded}

La columna \texttt{passenger\_count} se encuentra en formato númerico.
Tal y como visualizamos en el anterior punto existian valores ``raros''
en esta columna.

Vamos a eliminar todos los registros con \texttt{passenger\_count} igual
a 0, 7, 8 y 9.

\begin{Shaded}
\begin{Highlighting}[]
\NormalTok{train }\OtherTok{\textless{}{-}}\NormalTok{ train }\SpecialCharTok{\%\textgreater{}\%} \FunctionTok{filter}\NormalTok{(passenger\_count }\SpecialCharTok{!=} \DecValTok{0} \SpecialCharTok{\&}\NormalTok{ passenger\_count }\SpecialCharTok{!=} \DecValTok{7} \SpecialCharTok{\&}\NormalTok{ passenger\_count }\SpecialCharTok{!=} \DecValTok{8} \SpecialCharTok{\&}\NormalTok{ passenger\_count }\SpecialCharTok{!=} \DecValTok{9}\NormalTok{)}
\end{Highlighting}
\end{Shaded}

Las columnas \texttt{pickup\_*} y \texttt{dropoff\_*} continen las
coordenadas. En el apartado anterior de visualización creamos dos
columnas nuevas para mostrar la distancia entre el lugar de recogida y
el destino.

Por tanto, estas dos columnas no van a ser utilizas por lo que
procedemos a borrarlas.

\begin{Shaded}
\begin{Highlighting}[]
\NormalTok{train }\OtherTok{\textless{}{-}}\NormalTok{ train }\SpecialCharTok{\%\textgreater{}\%} \FunctionTok{select}\NormalTok{(}\SpecialCharTok{{-}}\NormalTok{pickup\_latitude)}
\NormalTok{train }\OtherTok{\textless{}{-}}\NormalTok{ train }\SpecialCharTok{\%\textgreater{}\%} \FunctionTok{select}\NormalTok{(}\SpecialCharTok{{-}}\NormalTok{pickup\_longitude)}
\NormalTok{train }\OtherTok{\textless{}{-}}\NormalTok{ train }\SpecialCharTok{\%\textgreater{}\%} \FunctionTok{select}\NormalTok{(}\SpecialCharTok{{-}}\NormalTok{dropoff\_latitude)}
\NormalTok{train }\OtherTok{\textless{}{-}}\NormalTok{ train }\SpecialCharTok{\%\textgreater{}\%} \FunctionTok{select}\NormalTok{(}\SpecialCharTok{{-}}\NormalTok{dropoff\_longitude)}
\end{Highlighting}
\end{Shaded}

Sin embargo, las dos columnas que creamos anteriormente presentan
registros con distancias igual a 0. Estas deben tratarse de errores por
lo que procedemos con su eliminación.

\begin{Shaded}
\begin{Highlighting}[]
\NormalTok{train }\OtherTok{\textless{}{-}}\NormalTok{ train }\SpecialCharTok{\%\textgreater{}\%} \FunctionTok{filter}\NormalTok{(distance\_euclidean }\SpecialCharTok{!=} \DecValTok{0} \SpecialCharTok{\&}\NormalTok{ distance\_manhattan }\SpecialCharTok{!=} \DecValTok{0}\NormalTok{)}
\end{Highlighting}
\end{Shaded}

Continuando con la última variable, como hemos pudimos observar tanto en
la visualización como con la función summary, la columna
\texttt{store\_and\_fwd\_flag} dispone de muy pocos valores de tipo Y, y
por tanto no nos aporta información.

Procedemos a borrarla.

\begin{Shaded}
\begin{Highlighting}[]
\NormalTok{train }\OtherTok{\textless{}{-}}\NormalTok{ train }\SpecialCharTok{\%\textgreater{}\%} \FunctionTok{select}\NormalTok{(}\SpecialCharTok{{-}}\NormalTok{store\_and\_fwd\_flag)}
\end{Highlighting}
\end{Shaded}

Para finalizar este apartado de preprocesamiento realizaremos un
normalizado de nuestras columnas para que tengan un valor comprendido
entre 0 y 1.

\begin{Shaded}
\begin{Highlighting}[]
\NormalTok{range\_model }\OtherTok{\textless{}{-}} \FunctionTok{preProcess}\NormalTok{(train, }\AttributeTok{method =} \StringTok{"range"}\NormalTok{)}
\NormalTok{train }\OtherTok{\textless{}{-}} \FunctionTok{predict}\NormalTok{(range\_model, }\AttributeTok{newdata =}\NormalTok{ train)}
\FunctionTok{rm}\NormalTok{(range\_model)}
\end{Highlighting}
\end{Shaded}

\hypertarget{predicciuxf3n}{%
\subsection{Predicción}\label{predicciuxf3n}}

\hypertarget{divisiuxf3n-del-conjunto}{%
\subsection{División del conjunto}\label{divisiuxf3n-del-conjunto}}

El primer paso es eliminar valores de nuestro dataset ya que debido a la
cantidad de registros no es posible su computación.

\begin{Shaded}
\begin{Highlighting}[]
\FunctionTok{set.seed}\NormalTok{(}\DecValTok{123}\NormalTok{)}
\NormalTok{ind }\OtherTok{\textless{}{-}} \FunctionTok{createDataPartition}\NormalTok{(train}\SpecialCharTok{$}\NormalTok{trip\_duration, }\AttributeTok{p =} \FloatTok{0.001}\NormalTok{, }\AttributeTok{list =} \ConstantTok{FALSE}\NormalTok{)}
\NormalTok{train }\OtherTok{\textless{}{-}}\NormalTok{ train[ind,]}
\FunctionTok{rm}\NormalTok{(ind)}
\end{Highlighting}
\end{Shaded}

Ahora dividiremos nuestro conjunto en train y test.

\begin{Shaded}
\begin{Highlighting}[]
\FunctionTok{set.seed}\NormalTok{(}\DecValTok{456}\NormalTok{)}
\NormalTok{ind }\OtherTok{\textless{}{-}} \FunctionTok{createDataPartition}\NormalTok{(train}\SpecialCharTok{$}\NormalTok{trip\_duration, }\AttributeTok{p =} \FloatTok{0.7}\NormalTok{, }\AttributeTok{list =} \ConstantTok{FALSE}\NormalTok{)}
\NormalTok{training\_set }\OtherTok{\textless{}{-}}\NormalTok{ train[ind,]}
\NormalTok{test\_set }\OtherTok{\textless{}{-}}\NormalTok{ train[}\SpecialCharTok{{-}}\NormalTok{ind,]}
\FunctionTok{rm}\NormalTok{(ind)}
\end{Highlighting}
\end{Shaded}

\hypertarget{importancia-de-las-variables}{%
\subsection{Importancia de las
variables}\label{importancia-de-las-variables}}

Vamos a realizar un entrenamiento mediante Random Forest para observar
cuales son las variables más importantes en nuestro dataset.

Se observa que las distancias son las más importantes, superando la
euclídea a la manhattan.

No vamos a eliminar ninguna de estas variables del dataset ya que vamos
a ir probando como se comportan los modelos con su eliminación.

\begin{Shaded}
\begin{Highlighting}[]
\NormalTok{model\_rf }\OtherTok{\textless{}{-}} \FunctionTok{train}\NormalTok{(trip\_duration }\SpecialCharTok{\textasciitilde{}}\NormalTok{ ., }\AttributeTok{data =}\NormalTok{ training\_set, }\AttributeTok{method =} \StringTok{"rf"}\NormalTok{)}
\NormalTok{imp }\OtherTok{\textless{}{-}} \FunctionTok{varImp}\NormalTok{(model\_rf)}
\FunctionTok{plot}\NormalTok{(imp, }\AttributeTok{main =} \StringTok{"Importancia variables (Random Forest)"}\NormalTok{)}
\end{Highlighting}
\end{Shaded}

\includegraphics{taxi_trip_fid_files/figure-latex/unnamed-chunk-28-1.pdf}

\hypertarget{entrenamiento}{%
\subsection{Entrenamiento}\label{entrenamiento}}

Vamos a entrenar nuestros modelos con el dataset completo, ya
preprocesado.

Empezaremos utilizando regresión lineal.

\begin{Shaded}
\begin{Highlighting}[]
\CommentTok{\# Dividir conjunto de train y test en dos, uno con cada distancia.}

\NormalTok{training\_set\_euclidean }\OtherTok{\textless{}{-}} \FunctionTok{select}\NormalTok{(training\_set, }\SpecialCharTok{{-}}\NormalTok{distance\_manhattan)}
\NormalTok{training\_set\_manhattan }\OtherTok{\textless{}{-}} \FunctionTok{select}\NormalTok{(training\_set, }\SpecialCharTok{{-}}\NormalTok{distance\_euclidean)}

\NormalTok{test\_set\_euclidean }\OtherTok{\textless{}{-}} \FunctionTok{select}\NormalTok{(test\_set, }\SpecialCharTok{{-}}\NormalTok{distance\_manhattan)}
\NormalTok{test\_set\_manhattan }\OtherTok{\textless{}{-}} \FunctionTok{select}\NormalTok{(test\_set, }\SpecialCharTok{{-}}\NormalTok{distance\_euclidean)}
\end{Highlighting}
\end{Shaded}

Ahora entrenamos el modelo con ambos datos de train:

\begin{Shaded}
\begin{Highlighting}[]
\NormalTok{model\_lm\_euclidean }\OtherTok{\textless{}{-}} \FunctionTok{train}\NormalTok{(trip\_duration }\SpecialCharTok{\textasciitilde{}}\NormalTok{ ., }
                            \AttributeTok{data =}\NormalTok{ training\_set\_euclidean, }
                            \AttributeTok{method =} \StringTok{"lm"}\NormalTok{)}
\NormalTok{predicted\_lm\_euclidean }\OtherTok{\textless{}{-}} \FunctionTok{predict}\NormalTok{(model\_lm\_euclidean, }
                                  \FunctionTok{select}\NormalTok{(test\_set\_euclidean, }\SpecialCharTok{{-}}\NormalTok{trip\_duration))}

\FunctionTok{postResample}\NormalTok{(predicted\_lm\_euclidean, test\_set\_euclidean}\SpecialCharTok{$}\NormalTok{trip\_duration)}
\end{Highlighting}
\end{Shaded}

\begin{verbatim}
##         RMSE     Rsquared          MAE 
## 0.0001128188 0.5936826823 0.0000828220
\end{verbatim}

\begin{Shaded}
\begin{Highlighting}[]
\NormalTok{model\_lm\_manhattan }\OtherTok{\textless{}{-}} \FunctionTok{train}\NormalTok{(trip\_duration }\SpecialCharTok{\textasciitilde{}}\NormalTok{ ., }
                            \AttributeTok{data =}\NormalTok{ training\_set\_manhattan, }
                            \AttributeTok{method =} \StringTok{"lm"}\NormalTok{)}
\NormalTok{predicted\_lm\_manhattan }\OtherTok{\textless{}{-}} \FunctionTok{predict}\NormalTok{(model\_lm\_manhattan, }
                                  \FunctionTok{select}\NormalTok{(test\_set\_manhattan, }\SpecialCharTok{{-}}\NormalTok{trip\_duration))}

\FunctionTok{postResample}\NormalTok{(predicted\_lm\_manhattan, test\_set\_manhattan}\SpecialCharTok{$}\NormalTok{trip\_duration)}
\end{Highlighting}
\end{Shaded}

\begin{verbatim}
##         RMSE     Rsquared          MAE 
## 0.0001127156 0.5929620046 0.0000825294
\end{verbatim}

\begin{Shaded}
\begin{Highlighting}[]
\NormalTok{resam }\OtherTok{\textless{}{-}} \FunctionTok{resamples}\NormalTok{(}\FunctionTok{list}\NormalTok{(}\AttributeTok{EUC =}\NormalTok{ model\_lm\_euclidean,}
                        \AttributeTok{MAN =}\NormalTok{ model\_lm\_manhattan))}
\FunctionTok{summary}\NormalTok{(resam)}
\end{Highlighting}
\end{Shaded}

\begin{verbatim}
## 
## Call:
## summary.resamples(object = resam)
## 
## Models: EUC, MAN 
## Number of resamples: 25 
## 
## MAE 
##             Min.      1st Qu.       Median         Mean      3rd Qu.
## EUC 7.470956e-05 7.934337e-05 8.173332e-05 8.183042e-05 8.430475e-05
## MAN 7.751571e-05 8.091833e-05 8.244996e-05 8.268354e-05 8.392152e-05
##             Max. NA's
## EUC 8.893851e-05    0
## MAN 9.077866e-05    0
## 
## RMSE 
##             Min.      1st Qu.       Median         Mean      3rd Qu.
## EUC 9.908712e-05 0.0001131512 0.0001202717 0.0001203689 0.0001312543
## MAN 1.090085e-04 0.0001145401 0.0001225887 0.0001221160 0.0001269971
##             Max. NA's
## EUC 0.0001383157    0
## MAN 0.0001435306    0
## 
## Rsquared 
##          Min.   1st Qu.    Median      Mean   3rd Qu.      Max. NA's
## EUC 0.5414675 0.5949269 0.6294260 0.6225607 0.6446965 0.7064989    0
## MAN 0.5158429 0.5884741 0.6139354 0.6133371 0.6589657 0.6885963    0
\end{verbatim}

Se puede observar que ambos dan resultados muy parecidos, aun así la
distancia euclídea da mejores resultados que la manhattan.

Vamos a volver a entrenar ambos modelos pero ahora realizando ajuste en
los hiperparámetros:

\begin{Shaded}
\begin{Highlighting}[]
\NormalTok{lm\_ctrl }\OtherTok{\textless{}{-}} \FunctionTok{trainControl}\NormalTok{(}
  \AttributeTok{method =} \StringTok{"repeatedcv"}\NormalTok{,}
  \AttributeTok{number =} \DecValTok{5}\NormalTok{,}
  \AttributeTok{repeats =} \DecValTok{5}\NormalTok{,}
  \AttributeTok{search =} \StringTok{"random"}
\NormalTok{)}

\NormalTok{model\_lm\_euclidean\_ctrl }\OtherTok{\textless{}{-}} \FunctionTok{train}\NormalTok{(trip\_duration }\SpecialCharTok{\textasciitilde{}}\NormalTok{ ., }
                            \AttributeTok{data =}\NormalTok{ training\_set\_euclidean, }
                            \AttributeTok{method =} \StringTok{"lm"}\NormalTok{,}
                            \AttributeTok{trControl =}\NormalTok{ lm\_ctrl,}
                            \AttributeTok{tuneLength =} \DecValTok{10}\NormalTok{)}

\NormalTok{predicted\_lm\_euclidean\_ctrl }\OtherTok{\textless{}{-}} \FunctionTok{predict}\NormalTok{(model\_lm\_euclidean, }
                                  \FunctionTok{select}\NormalTok{(test\_set\_euclidean, }\SpecialCharTok{{-}}\NormalTok{trip\_duration))}

\FunctionTok{postResample}\NormalTok{(predicted\_lm\_euclidean\_ctrl, test\_set\_euclidean}\SpecialCharTok{$}\NormalTok{trip\_duration)}
\end{Highlighting}
\end{Shaded}

\begin{verbatim}
##         RMSE     Rsquared          MAE 
## 0.0001128188 0.5936826823 0.0000828220
\end{verbatim}

\begin{Shaded}
\begin{Highlighting}[]
\NormalTok{model\_lm\_manhattan\_ctrl }\OtherTok{\textless{}{-}} \FunctionTok{train}\NormalTok{(trip\_duration }\SpecialCharTok{\textasciitilde{}}\NormalTok{ ., }
                            \AttributeTok{data =}\NormalTok{ training\_set\_manhattan, }
                            \AttributeTok{method =} \StringTok{"lm"}\NormalTok{,}
                            \AttributeTok{trControl =}\NormalTok{ lm\_ctrl,}
                            \AttributeTok{tuneLength =} \DecValTok{10}\NormalTok{)}

\NormalTok{predicted\_lm\_manhattan\_ctrl }\OtherTok{\textless{}{-}} \FunctionTok{predict}\NormalTok{(model\_lm\_manhattan, }
                                  \FunctionTok{select}\NormalTok{(test\_set\_manhattan, }\SpecialCharTok{{-}}\NormalTok{trip\_duration))}

\FunctionTok{postResample}\NormalTok{(predicted\_lm\_manhattan\_ctrl, test\_set\_manhattan}\SpecialCharTok{$}\NormalTok{trip\_duration)}
\end{Highlighting}
\end{Shaded}

\begin{verbatim}
##         RMSE     Rsquared          MAE 
## 0.0001127156 0.5929620046 0.0000825294
\end{verbatim}

\begin{Shaded}
\begin{Highlighting}[]
\NormalTok{resam }\OtherTok{\textless{}{-}} \FunctionTok{resamples}\NormalTok{(}\FunctionTok{list}\NormalTok{(}\AttributeTok{EUC =}\NormalTok{ model\_lm\_euclidean\_ctrl,}
                        \AttributeTok{MAN =}\NormalTok{ model\_lm\_manhattan\_ctrl))}
\FunctionTok{summary}\NormalTok{(resam)}
\end{Highlighting}
\end{Shaded}

\begin{verbatim}
## 
## Call:
## summary.resamples(object = resam)
## 
## Models: EUC, MAN 
## Number of resamples: 25 
## 
## MAE 
##             Min.      1st Qu.       Median         Mean      3rd Qu.
## EUC 7.270360e-05 7.802745e-05 8.044021e-05 8.115232e-05 8.560012e-05
## MAN 6.970988e-05 7.891683e-05 8.269996e-05 8.220356e-05 8.556611e-05
##             Max. NA's
## EUC 9.128758e-05    0
## MAN 9.092459e-05    0
## 
## RMSE 
##             Min.      1st Qu.       Median         Mean      3rd Qu.
## EUC 9.443334e-05 0.0001081502 0.0001171875 0.0001171842 0.0001276460
## MAN 9.444028e-05 0.0001051338 0.0001202042 0.0001189592 0.0001302339
##            Max. NA's
## EUC 0.000138404    0
## MAN 0.000152033    0
## 
## Rsquared 
##          Min.   1st Qu.    Median      Mean   3rd Qu.      Max. NA's
## EUC 0.5354619 0.5739745 0.6143498 0.6206134 0.6779628 0.6950297    0
## MAN 0.3758607 0.5472309 0.6237546 0.6090115 0.6766204 0.7409798    0
\end{verbatim}

Con el ajuste de los hiperparámetros podemos ver como la distancia
manhattan ha mejorado respecto a la euclídea.

Procedemos a entrenar el modelo utilizando XGBoost

\begin{Shaded}
\begin{Highlighting}[]
\NormalTok{training\_set\_factor }\OtherTok{\textless{}{-}} \FunctionTok{map\_df}\NormalTok{(training\_set, }\ControlFlowTok{function}\NormalTok{(columna) \{}
\NormalTok{  columna }\SpecialCharTok{\%\textgreater{}\%} 
    \FunctionTok{as.factor}\NormalTok{() }\SpecialCharTok{\%\textgreater{}\%} 
\NormalTok{    as.numeric }\SpecialCharTok{\%\textgreater{}\%} 
\NormalTok{    \{ . }\SpecialCharTok{{-}} \DecValTok{1}\NormalTok{ \}}
\NormalTok{\})}

\NormalTok{test\_set\_factor }\OtherTok{\textless{}{-}} \FunctionTok{map\_df}\NormalTok{(test\_set, }\ControlFlowTok{function}\NormalTok{(columna) \{}
\NormalTok{  columna }\SpecialCharTok{\%\textgreater{}\%} 
    \FunctionTok{as.factor}\NormalTok{() }\SpecialCharTok{\%\textgreater{}\%} 
\NormalTok{    as.numeric }\SpecialCharTok{\%\textgreater{}\%} 
\NormalTok{    \{ . }\SpecialCharTok{{-}} \DecValTok{1}\NormalTok{ \}}
\NormalTok{\})}

\NormalTok{train\_matrix }\OtherTok{\textless{}{-}} 
\NormalTok{  training\_set\_factor }\SpecialCharTok{\%\textgreater{}\%} 
  \FunctionTok{select}\NormalTok{(}\SpecialCharTok{{-}}\NormalTok{trip\_duration) }\SpecialCharTok{\%\textgreater{}\%} 
  \FunctionTok{as.matrix}\NormalTok{() }\SpecialCharTok{\%\textgreater{}\%} 
  \FunctionTok{xgb.DMatrix}\NormalTok{(}\AttributeTok{data =}\NormalTok{ ., }\AttributeTok{label =}\NormalTok{ training\_set\_factor}\SpecialCharTok{$}\NormalTok{trip\_duration)}

\NormalTok{test\_matrix }\OtherTok{\textless{}{-}} 
\NormalTok{  test\_set\_factor }\SpecialCharTok{\%\textgreater{}\%} 
  \FunctionTok{select}\NormalTok{(}\SpecialCharTok{{-}}\NormalTok{trip\_duration) }\SpecialCharTok{\%\textgreater{}\%} 
  \FunctionTok{as.matrix}\NormalTok{() }\SpecialCharTok{\%\textgreater{}\%} 
  \FunctionTok{xgb.DMatrix}\NormalTok{(}\AttributeTok{data =}\NormalTok{ ., }\AttributeTok{label =}\NormalTok{ test\_set\_factor}\SpecialCharTok{$}\NormalTok{trip\_duration)}

\NormalTok{param }\OtherTok{\textless{}{-}} \FunctionTok{list}\NormalTok{(}
      \AttributeTok{booster =} \StringTok{"gbtree"}\NormalTok{,}
      \AttributeTok{objective=}\StringTok{"reg:linear"}\NormalTok{,}
      \AttributeTok{eval\_metric =} \StringTok{"rmse"}\NormalTok{,}
      \AttributeTok{max\_depth =} \DecValTok{4}\NormalTok{,}
      \AttributeTok{eta =} \FloatTok{0.3}\NormalTok{,}
      \AttributeTok{subsample =} \FloatTok{0.8}\NormalTok{,}
      \AttributeTok{colsample\_bytree =} \FloatTok{0.8}
\NormalTok{      )}

\FunctionTok{set.seed}\NormalTok{(}\DecValTok{123}\NormalTok{)}
\NormalTok{model\_xgb }\OtherTok{\textless{}{-}} \FunctionTok{xgboost}\NormalTok{(}\AttributeTok{data =}\NormalTok{ train\_matrix,}
                     \AttributeTok{nrounds =} \DecValTok{10}\NormalTok{,}
                     \AttributeTok{params =}\NormalTok{ param,}
                     \AttributeTok{nthread =} \DecValTok{2}\NormalTok{)}
\end{Highlighting}
\end{Shaded}

\begin{verbatim}
## [17:40:03] WARNING: amalgamation/../src/objective/regression_obj.cu:203: reg:linear is now deprecated in favor of reg:squarederror.
## [1]  train-rmse:303.883722 
## [2]  train-rmse:229.012598 
## [3]  train-rmse:180.576932 
## [4]  train-rmse:150.687317 
## [5]  train-rmse:130.740838 
## [6]  train-rmse:119.400281 
## [7]  train-rmse:112.800194 
## [8]  train-rmse:108.174040 
## [9]  train-rmse:105.209131 
## [10] train-rmse:103.157635
\end{verbatim}

\begin{Shaded}
\begin{Highlighting}[]
\NormalTok{predict\_xgb }\OtherTok{\textless{}{-}} \FunctionTok{predict}\NormalTok{(model\_xgb, test\_matrix)}
\FunctionTok{postResample}\NormalTok{(predict\_xgb, test\_set\_factor}\SpecialCharTok{$}\NormalTok{trip\_duration)}
\end{Highlighting}
\end{Shaded}

\begin{verbatim}
##       RMSE   Rsquared        MAE 
## 76.0376857  0.5258062 60.9132809
\end{verbatim}

Se puede observar que los resultados obtenidos son peores que con la
regresión lineal.

Después de varias pruebas ajustando los hiperparámetros, y utilizando
validación cruzada, esos hiperparámetros son los que mejor resultado
aportan.

Para terminar con este modelo vamos a mostar una gráfica de la
importancia que han tenido las distintas variables.

\begin{Shaded}
\begin{Highlighting}[]
\NormalTok{xgb\_imp\_freq }\OtherTok{\textless{}{-}} \FunctionTok{xgb.importance}\NormalTok{(}\AttributeTok{feature\_names =} \FunctionTok{colnames}\NormalTok{(train\_matrix), }
                               \AttributeTok{model =}\NormalTok{ model\_xgb)}
\FunctionTok{xgb.plot.importance}\NormalTok{(xgb\_imp\_freq)}
\end{Highlighting}
\end{Shaded}

\includegraphics{taxi_trip_fid_files/figure-latex/unnamed-chunk-37-1.pdf}

En puntos anteriores entrenamos un random forest para determinar la
importancia de las variables.

Vamos a observar que resultados aporta:

\begin{Shaded}
\begin{Highlighting}[]
\NormalTok{predicted\_rf }\OtherTok{\textless{}{-}} \FunctionTok{predict}\NormalTok{(model\_rf, }
                          \FunctionTok{select}\NormalTok{(test\_set, }\SpecialCharTok{{-}}\NormalTok{trip\_duration))}

\FunctionTok{postResample}\NormalTok{(predicted\_rf, test\_set}\SpecialCharTok{$}\NormalTok{trip\_duration)}
\end{Highlighting}
\end{Shaded}

\begin{verbatim}
##         RMSE     Rsquared          MAE 
## 1.010730e-04 6.767166e-01 7.108522e-05
\end{verbatim}

Después de analizar los distintos modelos se llega a la conclusión de
que el mejor modelo para este conjunto de datos es la regresión lineal
utilizando la distancia Manhattan.

\begin{Shaded}
\begin{Highlighting}[]
\CommentTok{\# eliminación de variables usadas en supervisado}
\FunctionTok{rm}\NormalTok{(}\AttributeTok{list=}\FunctionTok{ls}\NormalTok{())}

\CommentTok{\# recarga de datos}
\NormalTok{train }\OtherTok{\textless{}{-}} \FunctionTok{read.csv}\NormalTok{(}\StringTok{"datos/train.csv"}\NormalTok{)}
\end{Highlighting}
\end{Shaded}

\hypertarget{no-supervisado}{%
\section{No supervisado}\label{no-supervisado}}

\hypertarget{transformaciuxf3n-de-datos}{%
\subsection{Transformación de datos}\label{transformaciuxf3n-de-datos}}

La fase de transformación implica la modificación y limpieza de datos
para que puedan ser utilizados posteriormente por el proceso DM (Data
Mining).

\hypertarget{cuxe1lculo-de-distancia}{%
\subsubsection{Cálculo de distancia}\label{cuxe1lculo-de-distancia}}

Se añade una columna en el dataset que represente la distancia entre el
punto de recogida y el de llegada. Para ello, se usará la función de
distancia Harvesine, una fórmula matemática utilizada para calcular la
distancia en una superficie esférica (en este caso de la tierra) entre
dos puntos geográficos (punto de recogida y punto de llegada). Estos
datos vienen dados en metros.

Se ha decidido usar esta distancia porque mide distancias rectilíneas de
forma similar a la euclídea (al menos para una superficie esférica como
es el planeta Tierra), y esta distancia supuso la variable más
importante en el preprocesamiento del apartado anterior.

\begin{Shaded}
\begin{Highlighting}[]
\CommentTok{\# calcular distancia Harversine}
\NormalTok{train }\OtherTok{\textless{}{-}}\NormalTok{ train }\SpecialCharTok{\%\textgreater{}\%}
  \FunctionTok{mutate}\NormalTok{(}\AttributeTok{distance =} \FunctionTok{distHaversine}\NormalTok{(}\FunctionTok{cbind}\NormalTok{(}\AttributeTok{longitude =}\NormalTok{ pickup\_longitude, }\AttributeTok{latitude =}\NormalTok{ pickup\_latitude), }\FunctionTok{cbind}\NormalTok{(}\AttributeTok{longitude =}\NormalTok{ dropoff\_longitude, }\AttributeTok{latitude =}\NormalTok{ dropoff\_latitude)))}

\CommentTok{\# representa las caracterícticas de distancia}
\FunctionTok{summary}\NormalTok{(train}\SpecialCharTok{$}\NormalTok{distance)}
\end{Highlighting}
\end{Shaded}

\begin{verbatim}
##    Min. 1st Qu.  Median    Mean 3rd Qu.    Max. 
##       0    1233    2096    3445    3880 1242299
\end{verbatim}

\hypertarget{eliminaciuxf3n-de-datos}{%
\subsubsection{Eliminación de datos}\label{eliminaciuxf3n-de-datos}}

Dado que el conjunto de datos es grande y existen columnas redundantes,
se va a eliminar aquellas no relevantes para simplificar y mejorar la
eficiencia del análisis.

\begin{Shaded}
\begin{Highlighting}[]
\CommentTok{\# eliminación columnas no relevantes}
\NormalTok{train}\SpecialCharTok{$}\NormalTok{id }\OtherTok{\textless{}{-}} \ConstantTok{NULL}
\NormalTok{train}\SpecialCharTok{$}\NormalTok{vendor\_id }\OtherTok{\textless{}{-}} \ConstantTok{NULL}
\NormalTok{train}\SpecialCharTok{$}\NormalTok{passenger\_count }\OtherTok{\textless{}{-}} \ConstantTok{NULL}
\NormalTok{train}\SpecialCharTok{$}\NormalTok{pickup\_datetime }\OtherTok{\textless{}{-}} \ConstantTok{NULL}
\NormalTok{train}\SpecialCharTok{$}\NormalTok{dropoff\_datetime }\OtherTok{\textless{}{-}} \ConstantTok{NULL}
\NormalTok{train}\SpecialCharTok{$}\NormalTok{store\_and\_fwd\_flag }\OtherTok{\textless{}{-}} \ConstantTok{NULL}
\NormalTok{train}\SpecialCharTok{$}\NormalTok{pickup\_longitude }\OtherTok{\textless{}{-}} \ConstantTok{NULL}
\NormalTok{train}\SpecialCharTok{$}\NormalTok{pickup\_latitude }\OtherTok{\textless{}{-}} \ConstantTok{NULL}
\NormalTok{train}\SpecialCharTok{$}\NormalTok{dropoff\_longitude }\OtherTok{\textless{}{-}} \ConstantTok{NULL}
\NormalTok{train}\SpecialCharTok{$}\NormalTok{dropoff\_latitude }\OtherTok{\textless{}{-}} \ConstantTok{NULL}

\CommentTok{\# se representan el dataset train}
\FunctionTok{head}\NormalTok{(train)}
\end{Highlighting}
\end{Shaded}

\begin{verbatim}
##   trip_duration distance
## 1           455 1500.199
## 2           663 1807.530
## 3          2124 6392.251
## 4           429 1487.163
## 5           435 1189.920
## 6           443 1100.174
\end{verbatim}

\hypertarget{datos-imperfectos}{%
\subsubsection{Datos imperfectos}\label{datos-imperfectos}}

A continuación, se va a representar la distancia frente a la duración
para visualizar si existen datos imperfectos.

\begin{Shaded}
\begin{Highlighting}[]
\CommentTok{\# REPRESENTACIÓN }\AlertTok{TODO}\CommentTok{ DATASET {-}\textgreater{} tarda en cargar}
\CommentTok{\# plot(x = train$distance, y = train$trip\_duration)}

\CommentTok{\# REPRESENTACIÓN ALEATORIA}
\NormalTok{temp\_train }\OtherTok{\textless{}{-}} \FunctionTok{sample\_n}\NormalTok{(train, }\DecValTok{10000}\NormalTok{)}
\FunctionTok{plot}\NormalTok{(}\AttributeTok{x =}\NormalTok{ temp\_train}\SpecialCharTok{$}\NormalTok{distance, }\AttributeTok{y =}\NormalTok{ temp\_train}\SpecialCharTok{$}\NormalTok{trip\_duration)}
\end{Highlighting}
\end{Shaded}

\includegraphics{taxi_trip_fid_files/figure-latex/unnamed-chunk-42-1.pdf}

\begin{Shaded}
\begin{Highlighting}[]
\FunctionTok{rm}\NormalTok{(temp\_train)}
\end{Highlighting}
\end{Shaded}

Tras la representación, puede apreciarse que existen datos aislados con
distancias de hasta 80km y con una duración de 16'6 horas. Lo cual nos
indica una incongruencia, ya que esta ciudad en su parte más ancha mide
13'4km.

Por esta razón, se va a acotar tanto la duración como la distancia en
unas cotas que representen gran parte del conjunto de los datos.

Para la \textbf{duración}, se va a probar manualmente a acotar la
duración con un intervalo máximo y mínimo. En este caso, se ha probado
con el intervalo {[}10, 10.000{]} segundos, descartando solamente un
0.29\% del total de los datos.

\begin{Shaded}
\begin{Highlighting}[]
\CommentTok{\# ACOTAR DURACIÓN}

\FunctionTok{summary}\NormalTok{(train}\SpecialCharTok{$}\NormalTok{trip\_duration)}
\end{Highlighting}
\end{Shaded}

\begin{verbatim}
##    Min. 1st Qu.  Median    Mean 3rd Qu.    Max. 
##       1     397     662     959    1075 3526282
\end{verbatim}

\begin{Shaded}
\begin{Highlighting}[]
\CommentTok{\# acotamos duración máxima}
\NormalTok{max\_dur }\OtherTok{\textless{}{-}} \DecValTok{10000}
\NormalTok{lim\_max\_dur }\OtherTok{\textless{}{-}} \FunctionTok{nrow}\NormalTok{(train }\SpecialCharTok{\%\textgreater{}\%} \FunctionTok{filter}\NormalTok{(trip\_duration }\SpecialCharTok{\textgreater{}=}\NormalTok{ max\_dur))}
\FunctionTok{sprintf}\NormalTok{(}\StringTok{"Límitando como máximo la duración a \%ss {-}\textgreater{} eliminación \%s elementos (\%s\%\% del total)"}\NormalTok{, max\_dur, lim\_max\_dur, }\DecValTok{100} \SpecialCharTok{*}\NormalTok{ lim\_max\_dur }\SpecialCharTok{/} \FunctionTok{nrow}\NormalTok{(train))}
\end{Highlighting}
\end{Shaded}

\begin{verbatim}
## [1] "Límitando como máximo la duración a 10000s -> eliminación 2123 elementos (0.145546137371422% del total)"
\end{verbatim}

\begin{Shaded}
\begin{Highlighting}[]
\CommentTok{\# acotamos duración mínima}
\NormalTok{min\_dur }\OtherTok{\textless{}{-}} \DecValTok{10}
\NormalTok{lim\_min\_dur }\OtherTok{\textless{}{-}} \FunctionTok{nrow}\NormalTok{(train }\SpecialCharTok{\%\textgreater{}\%} \FunctionTok{filter}\NormalTok{(trip\_duration }\SpecialCharTok{\textless{}=}\NormalTok{ min\_dur))}
\FunctionTok{sprintf}\NormalTok{(}\StringTok{"Límitando como mínimo la duración a \%ss {-}\textgreater{} eliminación \%s elementos (\%s\%\% del total)"}\NormalTok{, min\_dur, lim\_min\_dur, }\DecValTok{100} \SpecialCharTok{*}\NormalTok{ lim\_min\_dur }\SpecialCharTok{/} \FunctionTok{nrow}\NormalTok{(train))}
\end{Highlighting}
\end{Shaded}

\begin{verbatim}
## [1] "Límitando como mínimo la duración a 10s -> eliminación 2166 elementos (0.148494080803815% del total)"
\end{verbatim}

\begin{Shaded}
\begin{Highlighting}[]
\CommentTok{\# aplicando ambas restricciones}
\NormalTok{lim\_dur }\OtherTok{\textless{}{-}} \FunctionTok{nrow}\NormalTok{(train }\SpecialCharTok{\%\textgreater{}\%} \FunctionTok{filter}\NormalTok{(trip\_duration }\SpecialCharTok{\textless{}=}\NormalTok{ min\_dur }\SpecialCharTok{|}\NormalTok{ trip\_duration }\SpecialCharTok{\textgreater{}=}\NormalTok{ max\_dur)) }\CommentTok{\# nolint}
\FunctionTok{sprintf}\NormalTok{(}\StringTok{"Límitando duración entre [\%s, \%s] {-}\textgreater{} eliminación \%s elementos (\%s\%\% del total)"}\NormalTok{, min\_dur, max\_dur, lim\_dur, }\DecValTok{100} \SpecialCharTok{*}\NormalTok{ lim\_dur }\SpecialCharTok{/} \FunctionTok{nrow}\NormalTok{(train)) }\CommentTok{\# nolint}
\end{Highlighting}
\end{Shaded}

\begin{verbatim}
## [1] "Límitando duración entre [10, 10000] -> eliminación 4289 elementos (0.294040218175237% del total)"
\end{verbatim}

\begin{Shaded}
\begin{Highlighting}[]
\CommentTok{\# descartamos aquellos que no cumplan restricciones}
\NormalTok{train }\OtherTok{\textless{}{-}}\NormalTok{ train }\SpecialCharTok{\%\textgreater{}\%} \FunctionTok{filter}\NormalTok{(trip\_duration }\SpecialCharTok{\textgreater{}}\NormalTok{ min\_dur }\SpecialCharTok{\&}\NormalTok{ trip\_duration }\SpecialCharTok{\textless{}}\NormalTok{ max\_dur)}
\FunctionTok{rm}\NormalTok{(max\_dur, lim\_max\_dur, min\_dur, lim\_min\_dur, lim\_dur)}
\end{Highlighting}
\end{Shaded}

Para la \textbf{distancia}, se va a probar manualmente a acotar la
distancia con un intervalo máximo y mínimo. En este caso, se ha probado
con el intervalo {[}10, 25.000{]} metros, descartando solamente un
0.58\% del total de los datos (que ya han sido previamente filtrados en
la duración.

\begin{Shaded}
\begin{Highlighting}[]
\CommentTok{\# ACOTAR DISTANCIA}

\FunctionTok{summary}\NormalTok{(train}\SpecialCharTok{$}\NormalTok{distance)}
\end{Highlighting}
\end{Shaded}

\begin{verbatim}
##    Min. 1st Qu.  Median    Mean 3rd Qu.    Max. 
##       0    1236    2099    3448    3883 1242299
\end{verbatim}

\begin{Shaded}
\begin{Highlighting}[]
\CommentTok{\# acotamos distancia máxima}
\NormalTok{max\_dist }\OtherTok{\textless{}{-}} \DecValTok{25000}
\NormalTok{lim\_max\_dist }\OtherTok{\textless{}{-}} \FunctionTok{nrow}\NormalTok{(train }\SpecialCharTok{\%\textgreater{}\%} \FunctionTok{filter}\NormalTok{(distance }\SpecialCharTok{\textgreater{}=}\NormalTok{ max\_dist))}
\FunctionTok{sprintf}\NormalTok{(}\StringTok{"Límitando como máximo la distancia a \%sm {-}\textgreater{} eliminación \%s elementos (\%s\%\% del total)"}\NormalTok{, max\_dist, lim\_max\_dist, }\DecValTok{100} \SpecialCharTok{*}\NormalTok{ lim\_max\_dist }\SpecialCharTok{/} \FunctionTok{nrow}\NormalTok{(train))}
\end{Highlighting}
\end{Shaded}

\begin{verbatim}
## [1] "Límitando como máximo la distancia a 25000m -> eliminación 1399 elementos (0.0961938453816297% del total)"
\end{verbatim}

\begin{Shaded}
\begin{Highlighting}[]
\CommentTok{\# acotamos distancia mínima}
\NormalTok{min\_dist }\OtherTok{\textless{}{-}} \DecValTok{10}
\NormalTok{lim\_min\_dist }\OtherTok{\textless{}{-}} \FunctionTok{nrow}\NormalTok{(train }\SpecialCharTok{\%\textgreater{}\%} \FunctionTok{filter}\NormalTok{(distance }\SpecialCharTok{\textless{}=}\NormalTok{ min\_dist))}
\FunctionTok{sprintf}\NormalTok{(}\StringTok{"Límitando como mínimo la distancia a \%sm {-}\textgreater{} eliminación \%s elementos (\%s\%\% del total)"}\NormalTok{, min\_dist, lim\_min\_dist, }\DecValTok{100} \SpecialCharTok{*}\NormalTok{ lim\_min\_dist }\SpecialCharTok{/} \FunctionTok{nrow}\NormalTok{(train))}
\end{Highlighting}
\end{Shaded}

\begin{verbatim}
## [1] "Límitando como mínimo la distancia a 10m -> eliminación 7108 elementos (0.488738994262061% del total)"
\end{verbatim}

\begin{Shaded}
\begin{Highlighting}[]
\CommentTok{\# aplicando ambas restricciones}
\NormalTok{lim\_dist }\OtherTok{\textless{}{-}} \FunctionTok{nrow}\NormalTok{(train }\SpecialCharTok{\%\textgreater{}\%} \FunctionTok{filter}\NormalTok{(distance }\SpecialCharTok{\textless{}=}\NormalTok{ min\_dist }\SpecialCharTok{|}\NormalTok{ distance }\SpecialCharTok{\textgreater{}=}\NormalTok{ max\_dist))}
\FunctionTok{sprintf}\NormalTok{(}\StringTok{"Límitando distancia entre [\%s, \%s] {-}\textgreater{} eliminación \%s elementos (\%s\%\% del total)"}\NormalTok{, min\_dist, max\_dist, lim\_dist, }\DecValTok{100} \SpecialCharTok{*}\NormalTok{ lim\_dist }\SpecialCharTok{/} \FunctionTok{nrow}\NormalTok{(train))}
\end{Highlighting}
\end{Shaded}

\begin{verbatim}
## [1] "Límitando distancia entre [10, 25000] -> eliminación 8507 elementos (0.584932839643691% del total)"
\end{verbatim}

\begin{Shaded}
\begin{Highlighting}[]
\CommentTok{\# descartamos aquellos que no cumplan restricciones}
\NormalTok{train }\OtherTok{\textless{}{-}}\NormalTok{ train }\SpecialCharTok{\%\textgreater{}\%} \FunctionTok{filter}\NormalTok{(distance }\SpecialCharTok{\textgreater{}}\NormalTok{ min\_dist }\SpecialCharTok{\&}\NormalTok{ distance }\SpecialCharTok{\textless{}}\NormalTok{ max\_dist)}
\FunctionTok{rm}\NormalTok{(max\_dist, lim\_max\_dist, min\_dist, lim\_min\_dist, lim\_dist)}
\end{Highlighting}
\end{Shaded}

\hypertarget{normalizaciuxf3n-de-los-datos}{%
\subsubsection{Normalización de los
datos}\label{normalizaciuxf3n-de-los-datos}}

Para que los datos dispongan de una distribución similar y estén en la
misma escala se precisa realizar la normalización. Esta transformará
tanto la duración como la distancia entre un \textbf{intervalo {[}0,
1{]}}.

\begin{Shaded}
\begin{Highlighting}[]
\CommentTok{\# normalización de duración}
\NormalTok{train}\SpecialCharTok{$}\NormalTok{trip\_duration }\OtherTok{\textless{}{-}} \FunctionTok{rescale}\NormalTok{(train}\SpecialCharTok{$}\NormalTok{trip\_duration)}

\CommentTok{\# normalización de distancia}
\NormalTok{train}\SpecialCharTok{$}\NormalTok{distance }\OtherTok{\textless{}{-}} \FunctionTok{rescale}\NormalTok{(train}\SpecialCharTok{$}\NormalTok{distance)}

\CommentTok{\# muestra el conjunto de entrenamiento}
\FunctionTok{head}\NormalTok{(train)}
\end{Highlighting}
\end{Shaded}

\begin{verbatim}
##   trip_duration   distance
## 1    0.04452020 0.05963621
## 2    0.06537652 0.07193551
## 3    0.21187205 0.25541526
## 4    0.04191317 0.05911447
## 5    0.04251479 0.04721888
## 6    0.04331696 0.04362724
\end{verbatim}

\hypertarget{representaciuxf3n-de-datos}{%
\subsubsection{Representación de
datos}\label{representaciuxf3n-de-datos}}

Una vez se ha aplicado las distintas técnicas de transformación al
dataset, se va a representar para identificar a simple vista si existen
algunos datos incongruentes. Tras el análisis visual, se puede concretar
que el dataset se encuentra preparado para su uso en el proceso de Data
Mining.

\begin{Shaded}
\begin{Highlighting}[]
\CommentTok{\# REPRESENTACIÓN }\AlertTok{TODO}\CommentTok{ DATASET {-}\textgreater{} tarda en cargar}
\CommentTok{\# plot(x = train$distance, y = train$trip\_duration)}

\CommentTok{\# REPRESENTACIÓN ALEATORIA}
\NormalTok{temp\_train }\OtherTok{\textless{}{-}} \FunctionTok{sample\_n}\NormalTok{(train, }\DecValTok{50000}\NormalTok{)}
\FunctionTok{plot}\NormalTok{(}\AttributeTok{x =}\NormalTok{ temp\_train}\SpecialCharTok{$}\NormalTok{distance, }\AttributeTok{y =}\NormalTok{ temp\_train}\SpecialCharTok{$}\NormalTok{trip\_duration)}
\end{Highlighting}
\end{Shaded}

\includegraphics{taxi_trip_fid_files/figure-latex/unnamed-chunk-46-1.pdf}

\begin{Shaded}
\begin{Highlighting}[]
\FunctionTok{rm}\NormalTok{(temp\_train)}
\end{Highlighting}
\end{Shaded}

\hypertarget{data-mining-clustering}{%
\subsection{Data Mining: Clustering}\label{data-mining-clustering}}

Data Mining es el proceso de explorar y analizar grandes conjuntos de
datos con el fin de descubrir patrones y relaciones ocultos. Dentro de
este proceso, existen las técnicas de clustering, que son aquellas que
se utilizan para agrupar datos similares en ``clusters'', conjuntos o
grupos.

En este caso, se va a aplicar clustering para identificar los distintos
tipos de trayecto en función de la distancia y de la duración.

\hypertarget{clustering-con-k-means}{%
\subsubsection{Clustering con K-means}\label{clustering-con-k-means}}

\begin{enumerate}
\def\labelenumi{\arabic{enumi}.}
\tightlist
\item
  \textbf{Obtener número óptimo de clusters}
\end{enumerate}

Se representa la compactación en relación con el número de clusters.

\begin{Shaded}
\begin{Highlighting}[]
\NormalTok{vector\_compactacion }\OtherTok{\textless{}{-}} \DecValTok{0}
\ControlFlowTok{for}\NormalTok{(i }\ControlFlowTok{in} \DecValTok{1}\SpecialCharTok{:}\DecValTok{15}\NormalTok{) \{}
\NormalTok{  km\_train\_aux2 }\OtherTok{\textless{}{-}} \FunctionTok{kmeans}\NormalTok{(train,}\AttributeTok{center=}\NormalTok{i,}\AttributeTok{nstar=}\DecValTok{20}\NormalTok{)}
\NormalTok{  vector\_compactacion[i] }\OtherTok{\textless{}{-}}\NormalTok{ km\_train\_aux2}\SpecialCharTok{$}\NormalTok{tot.withinss}
\NormalTok{\}}

\FunctionTok{par}\NormalTok{(}\AttributeTok{mfrow =} \FunctionTok{c}\NormalTok{(}\DecValTok{1}\NormalTok{,}\DecValTok{1}\NormalTok{)) }
\FunctionTok{plot}\NormalTok{(}\DecValTok{1}\SpecialCharTok{:}\DecValTok{15}\NormalTok{, vector\_compactacion, }\AttributeTok{type =} \StringTok{"b"}\NormalTok{, }
     \AttributeTok{xlab =} \StringTok{"Numero de clusters"}\NormalTok{, }
     \AttributeTok{ylab =} \StringTok{"Compactacion"}\NormalTok{)}
\end{Highlighting}
\end{Shaded}

Se determina que el número óptimo de clusters son 3, ya que a partir de
ahí se estabiliza.

\begin{enumerate}
\def\labelenumi{\arabic{enumi}.}
\setcounter{enumi}{1}
\tightlist
\item
  \textbf{Aplicar función kmeans}
\end{enumerate}

Se extrae los clusters mediante la técnica de k-means o k-medias.

\begin{Shaded}
\begin{Highlighting}[]
\CommentTok{\# división en clústeres}
\NormalTok{km\_train }\OtherTok{\textless{}{-}} \FunctionTok{kmeans}\NormalTok{(train, }\AttributeTok{center =} \DecValTok{3}\NormalTok{, }\AttributeTok{nstar =} \DecValTok{20}\NormalTok{)}

\CommentTok{\# breve resumen resultados tras aplicar la técnica}
\FunctionTok{summary}\NormalTok{(km\_train)}
\end{Highlighting}
\end{Shaded}

\begin{verbatim}
##              Length  Class  Mode   
## cluster      1445848 -none- numeric
## centers            6 -none- numeric
## totss              1 -none- numeric
## withinss           3 -none- numeric
## tot.withinss       1 -none- numeric
## betweenss          1 -none- numeric
## size               3 -none- numeric
## iter               1 -none- numeric
## ifault             1 -none- numeric
\end{verbatim}

\begin{enumerate}
\def\labelenumi{\arabic{enumi}.}
\setcounter{enumi}{2}
\tightlist
\item
  \textbf{Representación de clústeres}
\end{enumerate}

Se representan los clústeres extraídos mediante k-means.

\begin{Shaded}
\begin{Highlighting}[]
\CommentTok{\# representación de los clusters}
\FunctionTok{ggplot}\NormalTok{(train, }\FunctionTok{aes}\NormalTok{(}\AttributeTok{x =}\NormalTok{ distance, }\AttributeTok{y =}\NormalTok{ trip\_duration, }\AttributeTok{color =} \FunctionTok{factor}\NormalTok{(km\_train}\SpecialCharTok{$}\NormalTok{cluster))) }\SpecialCharTok{+} \FunctionTok{geom\_point}\NormalTok{()}
\end{Highlighting}
\end{Shaded}

\includegraphics{taxi_trip_fid_files/figure-latex/unnamed-chunk-49-1.pdf}

\hypertarget{clustering-con-uxe1rboles-jeruxe1rquicos}{%
\subsubsection{Clustering con árboles
jerárquicos}\label{clustering-con-uxe1rboles-jeruxe1rquicos}}

Son algoritmos que se basan en construir un dendograma de forma que se
secciona por el número de clusters deseados.

\begin{enumerate}
\def\labelenumi{\arabic{enumi}.}
\tightlist
\item
  \textbf{Calcular matriz de distancias}
\end{enumerate}

Este algoritmo usa la matriz de distancias para crear el dendograma, es
por ello que en primera instancia será necesario obtener esta matriz.

\begin{Shaded}
\begin{Highlighting}[]
\NormalTok{points }\OtherTok{\textless{}{-}} \FunctionTok{c}\NormalTok{(train}\SpecialCharTok{$}\NormalTok{distance, train}\SpecialCharTok{$}\NormalTok{trip\_duration)}
\NormalTok{dist\_tree }\OtherTok{\textless{}{-}} \FunctionTok{dist}\NormalTok{(points)}
\FunctionTok{rm}\NormalTok{(points)}
\end{Highlighting}
\end{Shaded}

Como se puede apreciar, la cantidad de datos para construir un
dendograma es muy grande para los datos dados, concretamente de 31Tb de
datos.

\begin{Shaded}
\begin{Highlighting}[]
\CommentTok{\# existen 1445848 registros}
\FunctionTok{nrow}\NormalTok{(train)}
\end{Highlighting}
\end{Shaded}

\begin{verbatim}
## [1] 1445848
\end{verbatim}

Dado que existen 1445848 elementos en el dataset, se va a reducir este
número para crear una matriz de distancias que pueda ser analizada
computacionalmente a día de hoy. Para ello, se ha extraido 20000
elementos de forma aleatoria del dataset para llevar a cabo esta tarea.

\begin{Shaded}
\begin{Highlighting}[]
\CommentTok{\# número de elementos}
\CommentTok{\# menor tamaño {-}\textgreater{} menor consumo de memoria y tiempo de CPU}
\NormalTok{n\_dataset }\OtherTok{\textless{}{-}} \DecValTok{20000}

\CommentTok{\# reducción de los datos de forma aleatoria}
\CommentTok{\# ¡ADVERTENCIA! En cada ejecución se tomarán unos datos diferentes}
\NormalTok{red\_train }\OtherTok{\textless{}{-}} \FunctionTok{sample\_n}\NormalTok{(train, n\_dataset)}
\FunctionTok{rm}\NormalTok{(n\_dataset)}

\CommentTok{\# representar datos seleccionados}
\FunctionTok{plot}\NormalTok{(}\AttributeTok{x =}\NormalTok{ red\_train}\SpecialCharTok{$}\NormalTok{distance, }\AttributeTok{y =}\NormalTok{ red\_train}\SpecialCharTok{$}\NormalTok{trip\_duration)}
\end{Highlighting}
\end{Shaded}

\includegraphics{taxi_trip_fid_files/figure-latex/unnamed-chunk-52-1.pdf}

Se crea la matriz de distancias reducida con los 20k elementos. Aún así,
esta matriz ocupa en memoria 1,5GB.

\begin{Shaded}
\begin{Highlighting}[]
\CommentTok{\# construcción de matriz de distancias}
\NormalTok{dist\_matrix }\OtherTok{\textless{}{-}} \FunctionTok{dist}\NormalTok{(red\_train)}

\CommentTok{\# obtener tamaño de matriz de distancias {-}\textgreater{} RAM}
\FunctionTok{format}\NormalTok{(}\FunctionTok{object.size}\NormalTok{(dist\_matrix), }\AttributeTok{units =} \StringTok{"GB"}\NormalTok{)}
\end{Highlighting}
\end{Shaded}

\begin{verbatim}
## [1] "1.5 Gb"
\end{verbatim}

\begin{enumerate}
\def\labelenumi{\arabic{enumi}.}
\setcounter{enumi}{1}
\tightlist
\item
  \textbf{Aplicar función hclust}
\end{enumerate}

Se aplica la función que permite extraer el agrupamiento jerárquico en
un conjunto de datos.

\begin{Shaded}
\begin{Highlighting}[]
\CommentTok{\# aplicar hclust para obtener árbol}
\NormalTok{hc }\OtherTok{\textless{}{-}} \FunctionTok{hclust}\NormalTok{(dist\_matrix)}
\FunctionTok{rm}\NormalTok{(dist\_matrix)}
\FunctionTok{summary}\NormalTok{(hc)}
\end{Highlighting}
\end{Shaded}

\begin{verbatim}
##             Length Class  Mode     
## merge       39998  -none- numeric  
## height      19999  -none- numeric  
## order       20000  -none- numeric  
## labels          0  -none- NULL     
## method          1  -none- character
## call            2  -none- call     
## dist.method     1  -none- character
\end{verbatim}

\begin{enumerate}
\def\labelenumi{\arabic{enumi}.}
\setcounter{enumi}{2}
\tightlist
\item
  \textbf{Visualización del dendograma}
\end{enumerate}

Se representa el dendograma o árbol de clústeres. Como podrá observarse,
el dendograma dispone de una gran cantidad de niveles debido a la
cantidad de datos y sus posibles agrupaciones (20000 datos con una
altura de 19999 niveles).

\begin{Shaded}
\begin{Highlighting}[]
\FunctionTok{plot}\NormalTok{(hc)}
\end{Highlighting}
\end{Shaded}

\includegraphics{taxi_trip_fid_files/figure-latex/unnamed-chunk-55-1.pdf}

\begin{enumerate}
\def\labelenumi{\arabic{enumi}.}
\setcounter{enumi}{3}
\tightlist
\item
  \textbf{Obtener número óptimo de clusters}
\end{enumerate}

Una vez tenemos el dendograma, será necesario determinar cúantos números
de clusters deseamos para cortar el dendograma. Para ello vamos a
representar el TSS en relación al número de clusters, para así
determinar la mejor opción de forma visual.

Nota: esta función puede tardar varios minutos.

\begin{Shaded}
\begin{Highlighting}[]
\CommentTok{\# representa WSS {-} K (nº de clusters)}
\CommentTok{\# computacionalmente complejo {-}\textgreater{} \textasciitilde{}5 min}
\FunctionTok{fviz\_nbclust}\NormalTok{(red\_train, }\AttributeTok{FUN =}\NormalTok{ hcut, }\AttributeTok{method =} \StringTok{"wss"}\NormalTok{)}
\end{Highlighting}
\end{Shaded}

Se ha optado por elegir 4 clusters, ya que a partir de este se
estabiliza y no aporta más valor tener más clusters.

\begin{enumerate}
\def\labelenumi{\arabic{enumi}.}
\setcounter{enumi}{4}
\tightlist
\item
  \textbf{Corte del dendograma}
\end{enumerate}

Se corta el dendograma en los 4 clusters que se han determinado
previamente. Posteriormente se representa.

\begin{Shaded}
\begin{Highlighting}[]
\CommentTok{\# nº clusters seleccionados para corte dendograma}
\NormalTok{k }\OtherTok{\textless{}{-}} \DecValTok{4}

\CommentTok{\# dendograma cortado}
\NormalTok{cut\_dend }\OtherTok{\textless{}{-}} \FunctionTok{cutree}\NormalTok{(hc, }\AttributeTok{k =}\NormalTok{ k)}
\NormalTok{class\_red\_train }\OtherTok{\textless{}{-}} \FunctionTok{mutate}\NormalTok{(red\_train, }\AttributeTok{cluster =}\NormalTok{ cut\_dend)}
\FunctionTok{rm}\NormalTok{(k, cut\_dend, hc)}
\FunctionTok{head}\NormalTok{(class\_red\_train)}
\end{Highlighting}
\end{Shaded}

\begin{verbatim}
##   trip_duration   distance cluster
## 1    0.02366389 0.04261694       1
## 2    0.06236839 0.11181339       1
## 3    0.03228718 0.04136541       1
## 4    0.03098366 0.04194493       1
## 5    0.11370701 0.09907438       1
## 6    0.25609145 0.36149221       1
\end{verbatim}

\begin{enumerate}
\def\labelenumi{\arabic{enumi}.}
\setcounter{enumi}{5}
\tightlist
\item
  \textbf{Representación de clusters}
\end{enumerate}

Se representan los clusteres para los puntos datos, tras el corte
realizado en el dendograma.

\begin{Shaded}
\begin{Highlighting}[]
\CommentTok{\# representa dendograma cortado}
\FunctionTok{ggplot}\NormalTok{(class\_red\_train, }\FunctionTok{aes}\NormalTok{(}\AttributeTok{x =}\NormalTok{ distance, }\AttributeTok{y =}\NormalTok{ trip\_duration, }\AttributeTok{color =} \FunctionTok{factor}\NormalTok{(cluster))) }\SpecialCharTok{+} \FunctionTok{geom\_point}\NormalTok{()}
\end{Highlighting}
\end{Shaded}

\includegraphics{taxi_trip_fid_files/figure-latex/unnamed-chunk-58-1.pdf}

\hypertarget{clustering-basado-en-densidad}{%
\subsubsection{Clustering basado en
densidad}\label{clustering-basado-en-densidad}}

\begin{enumerate}
\def\labelenumi{\arabic{enumi}.}
\tightlist
\item
  \textbf{Matriz de distancias}
\end{enumerate}

Se obtiene la matriz de distancias de la misma forma que se realiza en
el clustering jerárquico.

\begin{Shaded}
\begin{Highlighting}[]
\CommentTok{\# número de elementos}
\CommentTok{\# menor tamaño {-}\textgreater{} menor consumo de memoria y tiempo de CPU}
\NormalTok{n\_dataset }\OtherTok{\textless{}{-}} \DecValTok{5000}

\CommentTok{\# reducción de los datos de forma aleatoria}
\CommentTok{\# ¡ADVERTENCIA! En cada ejecución se tomarán unos datos diferentes}
\NormalTok{red\_train }\OtherTok{\textless{}{-}} \FunctionTok{sample\_n}\NormalTok{(train, n\_dataset)}

\CommentTok{\# representar datos seleccionados}
\FunctionTok{plot}\NormalTok{(}\AttributeTok{x =}\NormalTok{ red\_train}\SpecialCharTok{$}\NormalTok{distance, }\AttributeTok{y =}\NormalTok{ red\_train}\SpecialCharTok{$}\NormalTok{trip\_duration)}
\end{Highlighting}
\end{Shaded}

\includegraphics{taxi_trip_fid_files/figure-latex/unnamed-chunk-59-1.pdf}

\begin{Shaded}
\begin{Highlighting}[]
\CommentTok{\# construcción de matriz de distancias}
\NormalTok{dist\_matrix }\OtherTok{\textless{}{-}} \FunctionTok{dist}\NormalTok{(red\_train)}

\CommentTok{\# obtener tamaño de matriz de distancias {-}\textgreater{} RAM}
\FunctionTok{format}\NormalTok{(}\FunctionTok{object.size}\NormalTok{(dist\_matrix), }\AttributeTok{units =} \StringTok{"GB"}\NormalTok{)}
\end{Highlighting}
\end{Shaded}

\begin{verbatim}
## [1] "0.1 Gb"
\end{verbatim}

\begin{enumerate}
\def\labelenumi{\arabic{enumi}.}
\setcounter{enumi}{1}
\tightlist
\item
  \textbf{Obtener EPS} Mediante la función kNNdist se puede determinar
  el EPS que se va a utilizar.
\end{enumerate}

\begin{Shaded}
\begin{Highlighting}[]
\CommentTok{\# select number of clusters}
\NormalTok{k }\OtherTok{\textless{}{-}} \DecValTok{2}

\CommentTok{\# original function}
\CommentTok{\# kNNdistplot(dist\_matrix, k)}

\CommentTok{\# custom function}
\NormalTok{knn\_dist }\OtherTok{\textless{}{-}} \FunctionTok{sort}\NormalTok{(}\FunctionTok{kNNdist}\NormalTok{(dist\_matrix, k))}
\FunctionTok{plot}\NormalTok{(knn\_dist, }\AttributeTok{type =} \StringTok{"l"}\NormalTok{, }\AttributeTok{ylab =} \FunctionTok{paste}\NormalTok{(k, }\StringTok{"{-}NN distance"}\NormalTok{, }\AttributeTok{sep =} \StringTok{""}\NormalTok{), }\AttributeTok{xlab =} \StringTok{"Points (sample) sorted by distance"}\NormalTok{, }\AttributeTok{ylim =} \FunctionTok{c}\NormalTok{(}\DecValTok{0}\NormalTok{, }\FloatTok{0.03}\NormalTok{))}
\FunctionTok{abline}\NormalTok{(}\AttributeTok{h =}\NormalTok{ .}\DecValTok{008}\NormalTok{, }\AttributeTok{col =} \StringTok{"red"}\NormalTok{)}
\end{Highlighting}
\end{Shaded}

\includegraphics{taxi_trip_fid_files/figure-latex/unnamed-chunk-60-1.pdf}

\begin{enumerate}
\def\labelenumi{\arabic{enumi}.}
\setcounter{enumi}{2}
\tightlist
\item
  \textbf{Aplicar función dbscan} Se aplica la función dbscan para
  categorizar en los diferentes clusters.
\end{enumerate}

\begin{Shaded}
\begin{Highlighting}[]
\CommentTok{\# minimum of points per cluster}
\NormalTok{min\_pts }\OtherTok{\textless{}{-}}\NormalTok{ n\_dataset }\SpecialCharTok{/} \DecValTok{100}

\CommentTok{\# eps: max. dist. entre 2 puntos para ser considerado vecino de otro}
\NormalTok{eps }\OtherTok{\textless{}{-}} \FloatTok{0.008}

\CommentTok{\# uso de la función dbscan}
\FunctionTok{set.seed}\NormalTok{(}\DecValTok{1234}\NormalTok{)}
\NormalTok{db }\OtherTok{\textless{}{-}} \FunctionTok{dbscan}\NormalTok{(dist\_matrix, eps, min\_pts)}
\FunctionTok{rm}\NormalTok{(n\_dataset, min\_pts)}
\NormalTok{db}
\end{Highlighting}
\end{Shaded}

\begin{verbatim}
## DBSCAN clustering for 5000 objects.
## Parameters: eps = 0.008, minPts = 50
## Using euclidean distances and borderpoints = TRUE
## The clustering contains 1 cluster(s) and 2135 noise points.
## 
##    0    1 
## 2135 2865 
## 
## Available fields: cluster, eps, minPts, dist, borderPoints
\end{verbatim}

\begin{enumerate}
\def\labelenumi{\arabic{enumi}.}
\setcounter{enumi}{3}
\tightlist
\item
  \textbf{Representación de clusters} Se representan los diferentes
  clústeres extraídos mediante densidad.
\end{enumerate}

\begin{Shaded}
\begin{Highlighting}[]
\CommentTok{\# representación de los clusters}
\FunctionTok{ggplot}\NormalTok{(red\_train, }\FunctionTok{aes}\NormalTok{(}\AttributeTok{x =}\NormalTok{ distance, }\AttributeTok{y =}\NormalTok{ trip\_duration, }\AttributeTok{color =} \FunctionTok{factor}\NormalTok{(db}\SpecialCharTok{$}\NormalTok{cluster))) }\SpecialCharTok{+} \FunctionTok{geom\_point}\NormalTok{()}
\end{Highlighting}
\end{Shaded}

\includegraphics{taxi_trip_fid_files/figure-latex/unnamed-chunk-62-1.pdf}

\hypertarget{interpretaciuxf3n-de-los-datos}{%
\subsection{Interpretación de los
datos}\label{interpretaciuxf3n-de-los-datos}}

\hypertarget{aux}{%
\section{AUX}\label{aux}}

\hypertarget{muxe9todo-de-reducciuxf3n-de-datos-auxiliar}{%
\subsection{Método de reducción de datos
auxiliar}\label{muxe9todo-de-reducciuxf3n-de-datos-auxiliar}}

Este método fue usado para intentar equilibrar la carga de datos del
dataset, pero disponía del problema que indicaba mayor concentración de
puntos justamente en el comienzo de cada uno de los segmentos de
distancia.

Consiste en segmentar los datos en 3 grupos (menores de 1500 metros,
comprendidos entre 1500 y 3000 metros y mayores de 3000) y se obtendrán
un subconjunto de estos de forma aleatoria de 1000 elementos.
Posteriormente, estos 3 conjuntos se unirán entre sí para dar lugar al
conjunto sobre el que se va a contruir el dendograma.

\begin{Shaded}
\begin{Highlighting}[]
\CommentTok{\# 1º conj.}
\NormalTok{set1 }\OtherTok{\textless{}{-}} \FunctionTok{sample\_n}\NormalTok{(train }\SpecialCharTok{\%\textgreater{}\%} \FunctionTok{filter}\NormalTok{(distance }\SpecialCharTok{\textless{}=} \FloatTok{0.3}\NormalTok{), }\DecValTok{1000}\NormalTok{)}
\FunctionTok{head}\NormalTok{(set1)}
\end{Highlighting}
\end{Shaded}

\begin{verbatim}
##   trip_duration   distance
## 1    0.02867743 0.05280946
## 2    0.09876667 0.12014701
## 3    0.04782914 0.09605358
## 4    0.06517598 0.12807523
## 5    0.09465557 0.09040491
## 6    0.10979645 0.09395822
\end{verbatim}

\begin{Shaded}
\begin{Highlighting}[]
\FunctionTok{summary}\NormalTok{(set1)}
\end{Highlighting}
\end{Shaded}

\begin{verbatim}
##  trip_duration         distance      
##  Min.   :0.002406   Min.   :0.00191  
##  1st Qu.:0.036699   1st Qu.:0.04704  
##  Median :0.058408   Median :0.07396  
##  Mean   :0.067401   Mean   :0.09238  
##  3rd Qu.:0.089567   3rd Qu.:0.12021  
##  Max.   :0.267121   Max.   :0.29937
\end{verbatim}

\begin{Shaded}
\begin{Highlighting}[]
\CommentTok{\# 2º conj.}
\NormalTok{set2 }\OtherTok{\textless{}{-}} \FunctionTok{sample\_n}\NormalTok{(train }\SpecialCharTok{\%\textgreater{}\%} \FunctionTok{filter}\NormalTok{(distance }\SpecialCharTok{\textgreater{}} \FloatTok{0.3} \SpecialCharTok{\&}\NormalTok{ distance }\SpecialCharTok{\textless{}} \FloatTok{0.6}\NormalTok{), }\DecValTok{1000}\NormalTok{)}
\FunctionTok{head}\NormalTok{(set2)}
\end{Highlighting}
\end{Shaded}

\begin{verbatim}
##   trip_duration  distance
## 1    0.14017848 0.3031358
## 2    0.17818109 0.3406996
## 3    0.10508373 0.3389650
## 4    0.08623283 0.3204308
## 5    0.09415422 0.4239517
## 6    0.15481801 0.3398516
\end{verbatim}

\begin{Shaded}
\begin{Highlighting}[]
\FunctionTok{summary}\NormalTok{(set2)}
\end{Highlighting}
\end{Shaded}

\begin{verbatim}
##  trip_duration        distance     
##  Min.   :0.03499   Min.   :0.3001  
##  1st Qu.:0.12764   1st Qu.:0.3426  
##  Median :0.16048   Median :0.3832  
##  Mean   :0.17401   Mean   :0.4056  
##  3rd Qu.:0.20733   3rd Qu.:0.4540  
##  Max.   :0.83947   Max.   :0.5988
\end{verbatim}

\begin{Shaded}
\begin{Highlighting}[]
\CommentTok{\# 3º conj.}
\NormalTok{set3 }\OtherTok{\textless{}{-}} \FunctionTok{sample\_n}\NormalTok{(train }\SpecialCharTok{\%\textgreater{}\%} \FunctionTok{filter}\NormalTok{(distance }\SpecialCharTok{\textgreater{}=} \FloatTok{0.6}\NormalTok{), }\DecValTok{1000}\NormalTok{)}
\FunctionTok{head}\NormalTok{(set3)}
\end{Highlighting}
\end{Shaded}

\begin{verbatim}
##   trip_duration  distance
## 1     0.5802667 0.7583583
## 2     0.3329991 0.8206158
## 3     0.1790835 0.6262418
## 4     0.1633410 0.8358126
## 5     0.2584979 0.8404432
## 6     0.2444600 0.7213209
\end{verbatim}

\begin{Shaded}
\begin{Highlighting}[]
\FunctionTok{summary}\NormalTok{(set3)}
\end{Highlighting}
\end{Shaded}

\begin{verbatim}
##  trip_duration        distance     
##  Min.   :0.07711   Min.   :0.6001  
##  1st Qu.:0.19242   1st Qu.:0.7231  
##  Median :0.24391   Median :0.8033  
##  Mean   :0.26418   Mean   :0.7844  
##  3rd Qu.:0.31483   3rd Qu.:0.8427  
##  Max.   :0.76286   Max.   :0.9953
\end{verbatim}

\begin{Shaded}
\begin{Highlighting}[]
\CommentTok{\# unir todos los conjuntos}
\NormalTok{red\_train }\OtherTok{\textless{}{-}} \FunctionTok{rbind}\NormalTok{(set1, set2, set3)}
\FunctionTok{rm}\NormalTok{(set1, set2, set3)}
\FunctionTok{head}\NormalTok{(red\_train)}
\end{Highlighting}
\end{Shaded}

\begin{verbatim}
##   trip_duration   distance
## 1    0.02867743 0.05280946
## 2    0.09876667 0.12014701
## 3    0.04782914 0.09605358
## 4    0.06517598 0.12807523
## 5    0.09465557 0.09040491
## 6    0.10979645 0.09395822
\end{verbatim}

\begin{Shaded}
\begin{Highlighting}[]
\FunctionTok{summary}\NormalTok{(red\_train)}
\end{Highlighting}
\end{Shaded}

\begin{verbatim}
##  trip_duration         distance      
##  Min.   :0.002406   Min.   :0.00191  
##  1st Qu.:0.084077   1st Qu.:0.12035  
##  Median :0.154517   Median :0.38317  
##  Mean   :0.168528   Mean   :0.42747  
##  3rd Qu.:0.229219   3rd Qu.:0.72306  
##  Max.   :0.839467   Max.   :0.99531
\end{verbatim}

Se representa esta distribución.

\begin{Shaded}
\begin{Highlighting}[]
\FunctionTok{plot}\NormalTok{(}\AttributeTok{x =}\NormalTok{ red\_train}\SpecialCharTok{$}\NormalTok{distance, }\AttributeTok{y =}\NormalTok{ red\_train}\SpecialCharTok{$}\NormalTok{trip\_duration)}
\end{Highlighting}
\end{Shaded}

\includegraphics{taxi_trip_fid_files/figure-latex/unnamed-chunk-64-1.pdf}

\hypertarget{bibliografuxeda}{%
\section{Bibliografía}\label{bibliografuxeda}}

\url{https://www.datacamp.com/tutorial/hierarchical-clustering-R}
\url{https://uc-r.github.io/hc_clustering}

\end{document}
